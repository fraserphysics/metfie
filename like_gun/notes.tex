\documentclass[]{article}
\usepackage{amsmath,amsfonts,afterpage}
\usepackage{showlabels}
\usepackage[pdftex]{graphicx,color}
\newcommand{\normal}[2]{{\cal N}(#1,#2)}
\newcommand{\normalexp}[3]{ -\frac{1}{2}
      (#1 - #2)^T #3^{-1} (#1 - #2) }
\newcommand{\La}{{\cal L}}
\newcommand{\fnom}{\tilde f}
\newcommand{\fhat}{\hat f}
\newcommand{\COST}{\cal C}
\newcommand{\LL}{{\cal L}}
\newcommand{\Prob}{\text{Prob}}
\newcommand{\field}[1]{\mathbb{#1}}
\newcommand\REAL{\field{R}}
\newcommand\Z{\field{Z}}
\newcommand{\partialfixed}[3]{\left. \frac{\partial #1}{\partial
      #2}\right|_#3}
\newcommand{\partiald}[2]{\frac{\partial #1}{\partial #2}}
\newcommand{\argmin}{\operatorname*{argmin}}
\newcommand{\argmax}{\operatorname*{argmax}}
\newcommand\norm[1]{\left|#1\right|}
\newcommand\bv{\mathbf{v}}
\newcommand\bt{\mathbf{t}}
\newcommand\Vfunc{\mathbb{V}}
\newcommand\Vt{\mathbf{V}}
\newcommand\vexp{V_{\rm{exp}}}
\newcommand\texp{T_{\rm{exp}}}
\newcommand\cf{c_f}
\newcommand\cv{c_v}
\newcommand\fbasis{b_f}
\newcommand\vbasis{b_v}
\newcommand\tsim{{\mathbf t}_{\rm{sim}}}
\newcommand\DVDf{\partiald{\Vt}{f}}
\newcommand\Lbb{\mathbb{L}}
\title{Notes EOS Estimation Code}

\author{Andrew M.\ Fraser}
\begin{document}
\maketitle

\section{Notation for and Implementation of Basic Functions}
\label{sec:basic}


Given an experimental sequence of measured pairs of velocity and time
values, $(\vexp, \texp)$ and an initial model, $\fnom$, I describe how
to estimate a new model, $\fhat$.

\subsection{Notation}
\label{sec:basic_notation}

\begin{description}
\item[$(\vexp, \texp)$:] An experimental sequence of measured pairs of
  velocity and time
\item[$\bv$:] A model sequence of velocities
\item[$\Vt$:] A model map from  times to velocities, eg, $\bv = \Vt(\texp)$
\item[$\tsim$:] A sequence of closely spaced sample times at which to record
  simulated position and velocity for constructing $\Vt$.
\item[$f$:] An EOS function
\item[$\Vfunc$:] An expensive procedure that maps an EOS function to a
  function that maps time to velocity with $\Vt = \Vfunc(f)$
\item[$\cv,\vbasis$:] Vectors of spline coefficients and basis functions
  that define a $\Vt$
\item[$\cf,\fbasis$:] Vector of spline coefficients and basis functions
  that define an EOS
\item[$\DVDf$:] The derivative of $\Vt$ with respect to $f$;
  represented by the matrix $\partiald{\cv}{\cf}$.  Very expensive to
  evaluate.
\end{description}

\subsection{Implementation}
\label{sec:basic_implementation}

\begin{description}
\item[$\Vfunc(f)$:] Run a simulation and record a sequence of
  velocities $\bv$ at times $\tsim$.  Then fit a
  spline to $(\bv, \bt)$ to obtain $\Vt\iff \{cv,\vbasis\}$.
\item[$\Vt$:] Call the spline evaluation method to get $\bv = \Vt(\texp)$
\item[$f$:] A spline with coefficients $\cf$ and basis functions
  $\fbasis$
\item[$\DVDf$:] Use $N(\cf)+1$ different sets of coefficients $\cf$ to
  get as many EOSs $f$ and evaluate $\Vfunc(f)$ for each.  Use these
  finite differences to get a matrix that approximates
  $\partiald{\cv}{\cf}$.
\end{description}

\section{Priors, Likelihood and Optimization}
\label{sec:opt}

I use Gaussians with constant diagonal covariances for priors and
likelihood as follows:
\begin{align}
\vexp,\texp | f &\sim \normal{\vexp-\Vt_f(\texp)}{\Sigma_v} \\
\cf &\sim \normal{\cf - c_{\fnom}}{\Sigma_{\cf}}
\end{align}
Up to irrelevant constants $K$ the log probabilities are:
\begin{align}
  L(\cf) & \equiv \log(K_2) + \log \left(\Prob(\cf) \right)\\
  & = \normalexp{\cf}{c_{\fnom}}{\Sigma_{\cf}}\\
  L(\Vt_f) & \equiv \log(K_1) + \log \left(\Prob(\vexp,\texp | f) \right)
  \nonumber \\
  & = \normalexp{\vexp}{\Vt_f(\texp)}{\Sigma_v} \\
  &= -\sum_{i=1}^{N_{\rm exp}} \frac{(\vexp[i] -
    \Vt_f(\texp[i]))^2}{2\sigma_v[i]}, \\
  & \text{and in terms of spline coefficients }\cv[j]\text{ and basis
    functions } \vbasis[j]\\
  \label{eq:vfsum}
  \Vt_f(\texp[i]) &= \sum_j \cv[j]\vbasis[j](\texp[i]).
\end{align}
In \eqref{eq:vfsum} the only term on the right that depends on the EOS
$f$ is $\cv[j]$.  If $f=\sum_k \cf[k] \fbasis[k]$ and $\delta=\sum_k d[k]
\fbasis[k]$, then
\begin{equation}
  \label{eq:taylor}
  \Vt_{f+\delta}(\texp[i]) = \sum_j \cv[j]\vbasis[j](\texp[i]) +
  \sum_j \sum_k \DVDf[j,k]d[k]\vbasis[j](\texp[i]) +\text{ HOT}.
\end{equation}

Given $f$ (in terms of $\cf$) and the functions $\Vt$, $\DVDf$, etc.\
that depend on it, an optimization step consists of ignoring the
\emph{higher order terms} (HOT) in \eqref{eq:taylor} and solving for
the vector $d$ that maximizes
\begin{equation}
  \label{eq:L}
  \Lbb(\cf,d) = L(c_{f+\delta}) +
  \Lbb(f+\delta) \equiv L(\Vt_{f+\delta}(\texp)).
\end{equation}
Differentiating, I find
\begin{align*}
  \frac{d \Lbb(\cf,d)}{d d} &= (c_{\fnom} - \cf - d)^T
  \Sigma^{-1}_{\cf} + \sum_{i=1}^{N_{\rm exp}}
  \frac{\Vt_{f+\delta}(\texp[i]) - \vexp[i] }{\sigma_v[i]}
  \frac{d \Vt_{f+\delta}(\texp[i])}{d d} \\
  \frac{d \Lbb(\cf,d)}{d d[k]} &= 
  \frac{c_{\fnom}[k] - \cf[k] - d[k]}{\sigma_{\cf}[k]} +
  \sum_{i=1}^{N_{\rm exp}}
  \frac{\Vt_{f+\delta}(\texp[i]) - \vexp[i] }{\sigma_v[i]}
  \sum_j \DVDf[j,k] \vbasis[j] (\texp[i])
\end{align*}


\end{document}

%%%---------------
%%% Local Variables:
%%% eval: (TeX-PDF-mode)
%%% eval: (setq ispell-personal-dictionary "./localdict")
%%% End:
