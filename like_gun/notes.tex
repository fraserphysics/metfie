\documentclass[12pt]{article}
\usepackage{amsmath,amsfonts,afterpage}
\usepackage{showlabels}
\usepackage[pdftex]{graphicx,color}
\newcommand{\normal}[2]{{\cal N}(#1,#2)}
\newcommand{\normalexp}[3]{ -\frac{1}{2}
      (#1 - #2)^T #3^{-1} (#1 - #2) }
\newcommand{\La}{{\cal L}}
\newcommand{\fnom}{\tilde f}
\newcommand{\fhat}{\hat f}
\newcommand{\COST}{\cal C}
\newcommand{\LL}{{\cal L}}
\newcommand{\Prob}{\text{Prob}}
\newcommand{\field}[1]{\mathbb{#1}}
\newcommand\REAL{\field{R}}
\newcommand\Z{\field{Z}}
\newcommand{\partialfixed}[3]{\left. \frac{\partial #1}{\partial
      #2}\right|_#3}
\newcommand{\partiald}[2]{\frac{\partial #1}{\partial #2}}
\newcommand{\argmin}{\operatorname*{argmin}}
\newcommand{\argmax}{\operatorname*{argmax}}
\newcommand\norm[1]{\left|#1\right|}
\newcommand\bv{\mathbf{v}}
\newcommand\bt{\mathbf{t}}
\newcommand\Vfunc{\mathbb{V}}
\newcommand\Vt{\mathbf{V}}
\newcommand\vexp{V_{\rm{exp}}}
\newcommand\texp{T_{\rm{exp}}}
\newcommand\cf{c_f}
\newcommand\cv{c_v}
\newcommand\fbasis{b_f}
\newcommand\vbasis{b_v}
\newcommand\tsim{{\mathbf t}_{\rm{sim}}}
\newcommand\DVDf{\partiald{\Vt}{f}}
\newcommand\Lbb{\mathbb{L}}
\title{Notes EOS Estimation Code}

\author{Andrew M.\ Fraser}
\begin{document}
\maketitle

\section{Notation for and Implementation of Basic Functions}
\label{sec:basic}


Given an experimental sequence of measured pairs of velocity and time
values, $(\vexp, \texp)$ and an initial model, $\fnom$, I describe how
to estimate a new model, $\fhat$.

\subsection{Notation}
\label{sec:basic_notation}

\begin{description}
\item[$(\vexp, \texp)$:] An experimental sequence of measured pairs of
  velocity and time
\item[$\bv$:] A sequence of model velocities
\item[$\Vt$:] A map from times to model velocities, eg, $\bv =
  \Vt(\texp)$
\item[$\tsim$:] A sequence of closely spaced sample times at which to record
  simulated position and velocity for constructing $\Vt$.
\item[$f$:] An EOS function
\item[$\Vfunc$:] An expensive procedure that maps an EOS function to a
  function that maps time to velocity with $\Vt = \Vfunc(f)$
\item[$\cv,\vbasis$:] Vectors of spline coefficients and basis functions
  that define a $\Vt$
\item[$\cf,\fbasis$:] Vector of spline coefficients and basis functions
  that define an EOS
\item[$D$] Matrix with elements $D[j,k] = \partiald{\cv[j]}{\cf[k]}$
\item[$B$] Matrix with elements $B[i,j] = \vbasis[j](\texp[i])$
\item[$\DVDf$:] The derivative of $\Vt$ with respect to $f$; very
  expensive to evaluate.  $\DVDf (\texp)$ is represented by the matrix
  $(BD)^T$.
\end{description}

\subsection{Implementation}
\label{sec:basic_implementation}

\begin{description}
\item[$\Vfunc(f)$:] Run a simulation and record a sequence of
  velocities $\bv$ at times $\tsim$.  Then fit a
  spline to $(\bv, \bt)$ to obtain $\Vt\iff \{cv,\vbasis\}$.
\item[$\Vt$:] Call the spline evaluation method to get $\bv = \Vt(\texp)$
\item[$f$:] A spline with coefficients $\cf$ and basis functions
  $\fbasis$
\item[$\DVDf$:] Evaluate $\Vfunc(f)$ for $N(\cf)+1$ different sets of
  coefficients $\cf$, and use finite differences to get a matrix with
  elements $D[j,k] = \partiald{\cv[j]}{\cf[k]}$.  Then
  \begin{equation*}
    \partiald{\Vt(T)}{\cf[k]} = \sum_j D[j,k] \vbasis[j](T).
  \end{equation*}
\end{description}

\section{Priors, Likelihood and Optimization}
\label{sec:opt}

\subsection{Minimum Squared Error}
\label{sec:minsq}

Before calculating the step for the full log a posteriori probability,
I calculate a least squares step (maximizing the likelihood if all
$\sigma_v[i]$ are equal to each other).  Thus, I wish to find $d$ that
minimizes
\begin{equation}
  \label{eq:ssq}
  \tilde S(d) = \sum_{i=1}^{N_{\rm exp}} (\vexp[i] - \Vt_{f+\delta}(\texp[i]))^2,
\end{equation}
where $f+\delta = \sum_k (\cf[k] + d[k])\fbasis[k]$, and 
\begin{equation}
  \label{eq:taylor}
  \Vt_{f+\delta}(\texp[i]) = \Vt_f(\texp[i]) +
  \sum_j \sum_k \partiald{\cv[j]}{\cf[k]}d[k]\vbasis[j](\texp[i])
  +\text{ HOT}.
\end{equation}
Letting $\epsilon[i]$ denote $\vexp[i] - \Vt_f(\texp[i])$ and dropping
HOT, I write
\begin{align*}
  S(d) &\equiv \tilde S(d) - \text{HOT} = \sum_i \left( \epsilon[i] -
  \sum_j \sum_k \partiald{\cv[j]}{\cf[k]}d[k]\vbasis[j](\texp[i])
  \right)^2\\
  &= (\epsilon - BDd)^T(\epsilon - BDd)
\end{align*}
and
\begin{align*}
  \partiald{S}{d[l]} &= -2 \sum_i \left( \epsilon[i] -
  \sum_j \sum_k \partiald{\cv[j]}{\cf[k]}d[k]\vbasis[j](\texp[i])
  \right)
  \sum_j \partiald{\cv[j]}{\cf[l]}\vbasis[j](\texp[i])\\
  \partiald{S}{d} &= -2\left(BD\right)^T\left(\epsilon - BDd\right).
\end{align*}
Thus I seek $\hat d$ that solves
\begin{equation}
  \label{eq:dhat}
  \left(BD\right)^T\epsilon = \left(BD\right)^T BD \hat d.
\end{equation}

\subsection{A Posteriori Probability}
\label{sec:app}

I use Gaussians with constant diagonal covariances for priors and
likelihood as follows:
\begin{align}
\vexp,\texp | f+\delta &\sim
\normal{\vexp-\Vt_{f+\delta}(\texp)}{\Sigma_v} \\
\cf+d &\sim \normal{\cf+d - c_{\fnom}}{\Sigma_f}
\end{align}
Up to irrelevant constants $K$ the log probabilities are:
\begin{align}
  L_c(\cf+d) & \equiv \log(K_2) + \log \left(\Prob(\cf+d) \right)\\
  & = \normalexp{\cf+d}{c_{\fnom}}{\Sigma_f}\\
  L_v(\Vt_{f+\delta}) & \equiv \log(K_1) + \log
  \left(\Prob(\vexp,\texp | f+\delta) \right)
  \nonumber \\
  & = \normalexp{\vexp}{\Vt_{f+\delta}(\texp)}{\Sigma_v} \\
  &= \normalexp{\epsilon}{BDd}{\Sigma_v}.
\end{align}

Given $f$ (in terms of $\cf$) and the functions $\Vt$, $\DVDf$, etc.\
that depend on it, an optimization step consists of ignoring the
\emph{higher order terms} (HOT) and solving for the vector $d$ that
maximizes
\begin{equation}
  \label{eq:L}
  \Lbb(\cf,d) = L_c(\cf + d) + L_v(\Vt_{f+\delta}(\texp)).
\end{equation}
Differentiating, I find
\begin{align*}
  \partiald{\Lbb(\cf,d)}{d} &= (c_{\fnom} - \cf - d)^T
  \Sigma^{-1}_f + (\epsilon - BDd)^T \Sigma_v^{-1}BD.
\end{align*}
Thus I seek $\hat d$ that solves
\begin{equation}
  \label{eq:dmap}
  (BD)^T\epsilon + \Sigma_f^{-1} (c_{\tilde f} - \cf) = 
  \left((BD)^T\Sigma_v^{-1}BD + \Sigma_f^{-1} \right) \hat d.
\end{equation}


\end{document}

%%%---------------
%%% Local Variables:
%%% eval: (TeX-PDF-mode)
%%% eval: (setq ispell-personal-dictionary "./localdict")
%%% End:
