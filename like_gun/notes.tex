\documentclass[11pt]{article}
\usepackage{amsmath,amsfonts,afterpage}
%\usepackage{showlabels}
\usepackage[pdftex]{graphicx,color}
\newcommand{\normal}[2]{{\cal N}(#1,#2)}
\newcommand{\normalexp}[3]{ -\frac{1}{2}
      (#1 - #2)^T #3^{-1} (#1 - #2) }
\newcommand{\La}{{\cal L}}
\newcommand{\fnom}{\tilde f}
\newcommand{\fhat}{\hat f}
\newcommand{\COST}{\cal C}
\newcommand{\LL}{{\cal L}}
\newcommand{\Prob}{\text{Prob}}
\newcommand{\field}[1]{\mathbb{#1}}
\newcommand\REAL{\field{R}}
\newcommand\Z{\field{Z}}
\newcommand{\partialfixed}[3]{\left. \frac{\partial #1}{\partial
      #2}\right|_#3}
\newcommand{\partiald}[2]{\frac{\partial #1}{\partial #2}}
\newcommand{\argmin}{\operatorname*{argmin}}
\newcommand{\argmax}{\operatorname*{argmax}}
\newcommand\norm[1]{\left|#1\right|}
\newcommand\bv{\mathbf{v}}
\newcommand\bt{\mathbf{t}}
\newcommand\Vfunc{\mathbb{V}}
\newcommand\Vt{\mathbf{V}}
\newcommand\vexp{V_{\rm{exp}}}
\newcommand\texp{T_{\rm{exp}}}
\newcommand\cf{c_f}
\newcommand\cv{c_v}
\newcommand\fbasis{b_f}
\newcommand\vbasis{b_v}
\newcommand\tsim{{\mathbf t}_{\rm{sim}}}
\newcommand\DVDf{\partiald{\Vt}{f}}
\newcommand\Lbb{\mathbb{L}}
\title{Notes on EOS Estimation Code}

\author{Andrew M.\ Fraser}
\begin{document}
\maketitle

\section{Notation for and Implementation of Basic Functions}
\label{sec:basic}

Given an experimental sequence of measured pairs of velocity and time
values, $(\vexp, \texp)$ and an initial model, $\fnom$, I describe how
to estimate a new model, $\fhat$.

\subsection{Notation}
\label{sec:basic_notation}

\begin{description}
\item[$(\vexp, \texp)$:] An experimental sequence of measured pairs of
  velocity and time
\item[$\bv$:] A sequence of model velocities
\item[$\Vt$:] A map from times to model velocities, eg, $\bv =
  \Vt(\texp)$
\item[$\tsim$:] A sequence of closely spaced sample times at which to record
  simulated position and velocity for constructing $\Vt$.
\item[$f$:] An EOS function
\item[$\Vfunc$:] An expensive procedure that maps an EOS function to a
  function that maps time to velocity with $\Vt = \Vfunc(f)$
\item[$\cv,\vbasis$:] Vectors of spline coefficients and basis functions
  that define a $\Vt$
\item[$\cf,\fbasis$:] Vector of spline coefficients and basis functions
  that define an EOS
\item[$D$] Matrix with elements $D[j,k] = \partiald{\cv[j]}{\cf[k]}$
\item[$B$] Matrix with elements $B[i,j] = \vbasis[j](\texp[i])$, ie,
  the exactly linear map from $c_v$ to simulated data.
\item[$\DVDf$:] The derivative of $\Vt$ with respect to $f$; very
  expensive to evaluate.  $\DVDf (\texp)$ is represented by the matrix
  $(BD)^T$.
\end{description}

\subsection{Implementation}
\label{sec:basic_implementation}

\begin{description}
\item[$\Vfunc(f)$:] Run a simulation and record a sequence of
  velocities $\bv$ at times $\tsim$.  Then fit a
  spline to $(\bv, \bt)$ to obtain $\Vt\iff \{cv,\vbasis\}$.
\item[$\Vt$:] Call the spline evaluation method to get $\bv = \Vt(\texp)$
\item[$f$:] A spline with coefficients $\cf$ and basis functions
  $\fbasis$
\item[$\DVDf$:] Evaluate $\Vfunc(f)$ for $N(\cf)+1$ different sets of
  coefficients $\cf$, and use finite differences to get a matrix with
  elements $D[j,k] = \partiald{\cv[j]}{\cf[k]}$.  Then
  \begin{equation*}
    \partiald{\Vt(T)}{\cf[k]} = \sum_j D[j,k] \vbasis[j](T).
  \end{equation*}
\end{description}

\section{Priors, Likelihood and Optimization}
\label{sec:opt}

\subsection{Minimum Squared Error}
\label{sec:minsq}

Before calculating the step for the full log a posteriori probability,
I calculate a least squares step (maximizing the likelihood if all
$\sigma_v[i]$ are equal to each other).  Thus, I wish to find $d$ that
minimizes
\begin{equation}
  \label{eq:ssq}
  \tilde S(d) = \sum_{i=1}^{N_{\rm exp}} (\vexp[i] - \Vt_{f+\delta}(\texp[i]))^2,
\end{equation}
where $f+\delta = \sum_k (\cf[k] + d[k])\fbasis[k]$, and 
\begin{equation}
  \label{eq:taylor}
  \Vt_{f+\delta}(\texp[i]) = \Vt_f(\texp[i]) +
  \sum_j \sum_k \partiald{\cv[j]}{\cf[k]}d[k]\vbasis[j](\texp[i])
  +\text{ HOT}.
\end{equation}
Letting $\epsilon[i]$ denote $\vexp[i] - \Vt_f(\texp[i])$ and dropping
HOT, I write
\begin{align*}
  S(d) &\equiv \tilde S(d) - \text{HOT} = \sum_i \left( \epsilon[i] -
  \sum_j \sum_k \partiald{\cv[j]}{\cf[k]}d[k]\vbasis[j](\texp[i])
  \right)^2\\
  &= (\epsilon - BDd)^T(\epsilon - BDd)
\end{align*}
and
\begin{align*}
  \partiald{S}{d[l]} &= -2 \sum_i \left( \epsilon[i] -
  \sum_j \sum_k \partiald{\cv[j]}{\cf[k]}d[k]\vbasis[j](\texp[i])
  \right)
  \sum_j \partiald{\cv[j]}{\cf[l]}\vbasis[j](\texp[i])\\
  \partiald{S}{d} &= -2\left(BD\right)^T\left(\epsilon - BDd\right).
\end{align*}
Thus I seek $\hat d$ that solves
\begin{equation}
  \label{eq:dhat}
  \left(BD\right)^T\epsilon = \left(BD\right)^T BD \hat d.
\end{equation}

\subsection{A Posteriori Probability}
\label{sec:app}

I use Gaussians with constant diagonal covariances for priors and
likelihood as follows:
\begin{align}
\vexp,\texp | f+\delta &\sim
\normal{\vexp-\Vt_{f+\delta}(\texp)}{\Sigma_v} \\
\cf+d &\sim \normal{\cf+d - c_{\fnom}}{\Sigma_f}
\end{align}
Up to irrelevant constants $K$ the log probabilities are:
\begin{align}
  L_c(\cf+d) & \equiv \log(K_2) + \log \left(\Prob(\cf+d) \right)\\
  & = \normalexp{\cf+d}{c_{\fnom}}{\Sigma_f}\\
  L_v(\Vt_{f+\delta}) & \equiv \log(K_1) + \log
  \left(\Prob(\vexp,\texp | f+\delta) \right)
  \nonumber \\
  & = \normalexp{\vexp}{\Vt_{f+\delta}(\texp)}{\Sigma_v} \\
  &= \normalexp{\epsilon}{BDd}{\Sigma_v}.
\end{align}

Given $f$ (in terms of $\cf$) and the functions $\Vt$, $\DVDf$, etc.\
that depend on it, an optimization step consists of ignoring the
\emph{higher order terms} (HOT) and solving for the vector $d$ that
maximizes
\begin{equation}
  \label{eq:L}
  \Lbb(\cf,d) = L_c(\cf + d) + L_v(\Vt_{f+\delta}(\texp)).
\end{equation}
Differentiating, I find
\begin{align*}
  \partiald{\Lbb(\cf,d)}{d} &= (c_{\fnom} - \cf - d)^T
  \Sigma^{-1}_f + (\epsilon - BDd)^T \Sigma_v^{-1}BD.
\end{align*}
Thus I seek $\hat d$ that solves
\begin{equation}
  \label{eq:dmap}
  (BD)^T\Sigma_v^{-1}\epsilon + \Sigma_f^{-1} (c_{\tilde f} - \cf) = 
  \left((BD)^T\Sigma_v^{-1}BD + \Sigma_f^{-1} \right) \hat d.
\end{equation}
\subsection{Numerical Results}
\label{sec:numerical-results}

\newcommand{\freq}{k} %
Figures \ref{fig:d} and \ref{fig:fve} illustrate iterative maximum
likelihood fitting of an EOS.  The procedure starts with the following
initial function
\begin{equation}
  \label{eq:initial}
  \fnom(x) = \frac{C}{x^3},
\end{equation}
and the iterations approach the \emph{actual} function
\begin{align}
  \label{eq:actual}
  f(x) &= \fnom(x) + \frac{2 e^{-\frac{x^2}{2w^2}}
         \fnom(x_0)}{\freq}  \sin(\freq x)\\
   \text{ where }C &= 2.56\times 10^{10} \nonumber \\
  \freq &= 0.2 \nonumber \\
  w &= 0.2, \nonumber
\end{align}
which was used to generate the simulated data.
\begin{figure}
  \centering
    \resizebox{\columnwidth}{!}{\includegraphics{fig_d.pdf}}  
    \caption{Illustration of the derivative of the velocity function
      with respect to the pressure function, $\frac{d v}{d p}$.  The
      second and third columns correspond variations of the
      coefficients of basis functions that differ by a factor of ten.
      The lower left plot indicates that nonlinear part of the
      response is a factor of a thousand smaller than the linear
      response.  }
  \label{fig:d}
\end{figure}
\begin{figure}
  \centering
    \resizebox{0.8\columnwidth}{!}{\includegraphics{fig_fve.pdf}}  
    \caption{Maximum likelihood estimation of $f$.  The \emph{true}
      EOS appears as \emph{actual} in the upper plot, and the
      optimization starts with the \emph{initial} and ends with
      \emph{fit}.  The corresponding velocity as a function of
      position appears in the middle plot, and the sequence of errors
      in the velocity time series for each step in the optimization
      appears in the lower plot. }
  \label{fig:fve}
\end{figure}

\section{Comparing Model Classes}
\label{sec:classes}

We want to test the hypothesis that splines are better for building
models of an EOS than polynomials.  We hope to compare the bias and
variance of two sequences of estimators, one that uses splines and one
that uses polynomials.  We will study estimators for quantities of
interest that are functionals of the EOS that differ from the
functionals that correspond to the measurements.  In previous work, we
have analyzed the functionals for muzzle time and muzzle velocity for
an ideal gun, and in the sections above we have described an analysis
of velocity measurements that depend on time.  We hope that those
quantities of interest and those measurements are sufficiently
different in that the measurements are given as functions of time and
the quantities of interest are required as functions of position.

Here is the agenda:
\begin{enumerate}
\item Design two classes of EOS models, ie, pressure as a function of
  volume along an isentrope.  One class will be splines with $n$ knots
  equally spaced on a log scale, and the other will be $n^{\text{th}}$
  order polynomials.
\item Write code that draws realizations of experimental data based on
  an EOS that is outside of both model classes for any finite $n$.
\item Write code that estimates the coefficients of polynomial models
  of the EOS.
\item Calculate (or estimate with simulations) sequences (indexed by
  $n$) of bias and variance pairs of estimators for quantities of
  interest using the two classes of models.  We hope to find that the
  curve defined by the pairs for splines lies below the curve defined
  by pairs for polynomials.
\end{enumerate}

\subsection{Alternative Measurements}
\label{sec:alternative}

If we need measurements that are more different from the quantities of
interest, we have could use the following simplification of
measurements of cylinders.  Suppose that all of the material in a long
cylinder detonates in an instant and that the density after that is
uniform inside.  Thus the function $\left\{r(t): t>0 \right\}$ is a
sufficient specification of an experimental result.  Letting
$r(0) = r_0$ denote the initial radius the work done by the gas in
expanding to $r$ is
\begin{equation*}
  U(r) = \int_{r_0}^r p(v(r)) \frac{dv}{dr} dr = \int_{r_0}^r p(\pi
  r^2) 2\pi r dr.
\end{equation*}
That work is used to deform and accelerate the cladding.  We say that
the energy of deformation is
\begin{equation*}
  D(r) = \sigma \log \left(\frac{r}{r_0} \right),
\end{equation*}
and the kinetic energy of the cladding is
\begin{equation*}
  K = \frac{1}{2} m \left( \dot r \right)^2.
\end{equation*}
Thus the velocity as a function of radius is
\begin{equation*}
  v(r) = \sqrt{\frac{2}{m} \left( U(r) - \sigma \log
      \left(\frac{r}{r_0} \right) \right) },
\end{equation*}
and we can obtain the velocity as a function of time by numerical
integration.

\section*{Appendix}
\label{sec:appendix}

See other information in cmfSuite/doc/HE.tex

One may fetch source code for this project from google code and build
this document as follows:
\begin{verbatim}
->git clone https://code.google.com/p/metfie
->cd metfie/like_gun
->make notes.pdf
\end{verbatim}

\end{document}

%%%---------------
%%% Local Variables:
%%% eval: (TeX-PDF-mode)
%%% eval: (setq ispell-personal-dictionary "./localdict")
%%% End:
