\documentclass[12pt]{article} \usepackage{amsmath,amsfonts}
\usepackage{showlabels}
\usepackage[pdftex]{graphicx,color}

\newcommand{\htop}{{h_{\text{top}}}}
\newcommand{\T}{{\cal P}}
\newcommand{\logprob}{{\cal L}}
\newcommand{\measure}{\mu}
\newcommand{\Aop}{{\cal A}}
\newcommand{\Aindicate}{\alpha}
\newcommand{\slope}{s}
\newcommand{\density}{p}
\newcommand{\stationary}{\mu}
\newcommand{\function}{\phi}
\newcommand{\rightfunction}{{_{_{\scalebox{.4}{R}}} \rho}}
\newcommand{\leftfunction}{{_{_{\scalebox{.4}{L}}} \rho}}
\renewcommand{\th}{^{\text{th}}}
\newcommand{\field}[1]{\mathbb{#1}}
\newcommand\REAL{\field{R}}
\newcommand\G{\field{G}}
\newcommand\I{{\cal I}}

\title{Uniform Probability for a Continuous Time Monotonic Process }

\author{Andrew M.\ Fraser}
\begin{document}
\maketitle
\begin{abstract}
  I describe a probability measure for functions of time that are
  constrained both to lie in within certain bounds and to be
  monotonic.  In coordinates that make the constraints invariant with
  respect to shifts in time, the measure corresponds to a stationary
  stochastic process.  Roughly, the measure makes all allowed
  sequences have the same probability %FixMe: Does this make senses?
  with respect to a norm.
\end{abstract}

\section*{To Do:}
\label{sec:do}

\begin{itemize}
\item Understand behavior of eigenvalues and eigenfunctions in
  $\tau \rightarrow 0$ limit.  Include results of numerical
  calculations (with $n_g=3000$) in plots of eigenvalues and
  eigenvectors.
\item Figure~\ref{fig:paths}: Plot of
  $\tilde T_{t=\frac{1}{4},g_0=\frac{1}{2},n}$ for several values of
  $n$.
\item Surface plot of $p(g(t_1),g(t_0))$ for $t_0=0$ and
  $t_t=\frac{1}{4}$.
\end{itemize}

\section{Introduction}
\label{sec:introduction}
\begin{description}
\item[Motivation:] Priors for regions in function space constrained by
  physical laws.  Cite Hixson2000, Fritz1996,
  Pemberton2011LA-UR-11-04999, and FraserLA-UR-13-21824
\item[Connection to Information Theory:] Theorem 8 of Shannon.  Define
  stationary and Markov for discrete time discrete value process.
\item[Connection to Statistical Mechanics:] Jaynes maximum entropy
  derivation of Maxwell Boltzmann distribution.  Coordinate dependent.
\item[Connection to path integrals:]
\item[Not Levy process:] Connection to Brownian motion
\item[Outline:]
\end{description}

\section{Specific Problem}
\label{sec:specific}

I seek a \emph{uniform} distribution on a \emph{restricted} set of
functions.  Figure~\ref{fig:bounds} illustrates the set which I define
by the following constraints:
\begin{subequations}
  \label{eq:f_def}
  \begin{description}
  \item[Bounded:] For all values of the independent variable $t$ the
    function lies between $1-t$ and $-t$,
    \begin{equation}
      \label{eq:bounded_f}
      -t \leq f(t) \leq 1-t
    \end{equation}
  \item[Monotonic:] The function is monotonic,
    \begin{equation}
      \label{eq:monotonic_f}
      f(t+\delta) \leq f(t) ~ \forall (t,\delta): \delta \geq 0
    \end{equation}
  \end{description}
\end{subequations}

\begin{figure*}
  \centering
  \resizebox{0.55\textwidth}{!}{\includegraphics{bounds.pdf}}
  \caption{Bounds on allowed functions.  The upper plot illustrates
    the bounds given by \eqref{eq:f_def} and the middle plot
    illustrates the same bounds translated to the constraining
    equations \eqref{eq:g_def}.  In each the green trace is the upper
    limit, the blue trace is the lower limit and the red trace is an
    allowed function.  For $\delta=0.2$, allowed pairs $g(t),
    g(t+\delta)$ are indicated by the shaded region in the lower
    plot.}
  \label{fig:bounds}
\end{figure*}
If one derives a new function $g$ by shifting $f$ by $t$, ie,
\begin{equation}
  \label{eq:shift}
  g(t) = f(t) + t,
\end{equation}
then the constraints \eqref{eq:f_def} on $f$ imply the following
constraints\footnote{These constraints are
  \emph{invariant} with respect to shifts of the origin of time, ie,
  given two functions $g:\REAL \mapsto [0,1]$ and $\tilde g:\REAL
  \mapsto [0,1]$ with
  \begin{equation*}
    \tilde g(t) = g(t+\tau)
  \end{equation*}
  for a fixed value of $\tau$, $\tilde g$ satisfies the constraints
  \eqref{eq:g_def} iff $g$ satisfies the constraints.} on $g$:
\begin{subequations}
  \label{eq:g_def}
  \begin{align}
    \label{eq:bounded_g}
    0 &\leq g(t) \leq 1 ~ \forall t\\
    \label{eq:monotonic_g}
    g(t+\delta) &\leq g(t)+\delta ~ \forall(t,\delta): \delta \geq 0  
  \end{align}
\end{subequations}

While Eqns.~\eqref{eq:g_def} describe the constraints pretty simply,
the goal of this paper is the more interesting task of defining a
uniform distribution over the set.  I require that the probability
measure $\measure$ have the following properties:
\begin{description}
\item[Stationary:] If two sets of measurable functions $A$ and $B$
  differ only by a shift in time, ie, there is a fixed shift $\tau$
  such that $a \in A ~ \iff ~ \exists b \in B$ with $b(t) =
  a(t+\tau)~\forall t$, then $\mu(A) = \mu(B)$.
\item[Consistent:] (From
  https://en.wikipedia.org/wiki/Kolmogorov\_extension\_theorem) Let T
  denote some interval (thought of as "time"), and let $n \in
  \mathbb{N}$. For each $k \in \mathbb{N}$ and finite sequence of
  times $t_{1}, \dots, t_{k} \in T, let \nu_{t_{1} \dots t_{k}}$ be a
  probability measure on $(\mathbb{R}^{n})^{k}$. Suppose that these
  measures satisfy two consistency conditions:

  1. for all permutations $\pi$ of $\{ 1, \dots, k \}$ and measurable sets
  $F_{i} \subseteq \mathbb{R}^{n}$,

  $\nu_{t_{\pi (1)} \dots t_{\pi (k)}} \left( F_{\pi (1)} \times \dots
    \times F_{ \pi(k)} \right) = \nu_{t_{1} \dots t_{k}} \left( F_{1}
    \times \dots \times F_{k} \right)$;

  2. for all measurable sets $F_{i} \subseteq \mathbb{R}^{n},m \in
  \mathbb{N}$

  $\nu_{t_{1} \dots t_{k}} \left( F_{1} \times \dots \times F_{k}
  \right) = \nu_{t_{1} \dots t_{k} t_{k + 1}, \dots , t_{k+m}} \left(
    F_{1} \times \dots \times F_{k} \times \mathbb{R}^{n} \times \dots
    \times \mathbb{R}^{n} \right)$.
\end{description}

\section{Counting Paths}
\label{sec:counting}

\subsection{Notation}
\label{sec:notation1}

\begin{description}
\item[$G \subset [0,1)$] A single interval that is a subset of the
  unit interval.
\item[$G_{i,n}$] The interval $\left[ \frac{i}{n}, \frac{i+1}{n}
  \right)$.
\item[$\G_n$] Set of uniform sub-intervals between 0 and 1
  \begin{equation*}
    \G_n \equiv \left\{ G_{i,n} : 0 \leq i \leq n-1 \right\}
  \end{equation*}
\item[$\tau$] The time between samples.  In \eqref{eq:monotonic_g} I
  used $\delta$ to denote the time between samples.  I use a different
  symbol here so that I can consider $\delta$ as a finite time that is
  subdivided into smaller intervals of length $\tau$
\item[An allowed sequential pair] The pair $(G_i, G_j)$ is allowed if
  and only if there is a $g_0\in G_i$ and a $g_1 \in G_j$ such that
  for $\tau=\delta$ the constraints \eqref{eq:g_def} are satisfied by
  $g_0=g(t), g_1=g(t+\tau)$.
\item[$A_{n,\tau}$] The adjacency matrix of allowed pairs
  \begin{equation*}
    A_\tau[i,j] =
    \begin{cases}
      1 &\text{ if } (G_{i,n}, G_{j,n}) \text{ is an allowed pair}\\
      0 &\text{ otherwise}
    \end{cases}
  \end{equation*}
\item[$i_0^n$] The sequence of integers $\left( i_0, i_1, \ldots i_n
  \right)$.
\item[$\I(i_0^m,n,\tau)$] The function that \emph{indicates} allowed
  sequences:
  \begin{equation*}
    \I(i_0^m,n,\tau) =
    \begin{cases}
      1 & \text{if }A_{n,\tau}[i_k,i_{k+1}] = 1,~ \forall k \in [0,m-1]\\
      0 & \text{otherwise}
    \end{cases}
  \end{equation*}
\item[$N(i,j,m,n,\tau)$] The number of allowed sequences with $m$
  transitions that start with $i$ and end with $j$:
  \begin{equation*}
    N(i,j,m,n,\tau) = \left( A_{n,\tau} \right)^m[i,j].
  \end{equation*}
\item[$i(g,n)$] The partition element that contains $g$, ie,
  $g \in G_{i(g,n),n}$.
\item[$P(G|f,h, \delta)$] The probability that $g(t+\delta) \in G$
  given $g(t)=f$ and $g(t+2\delta)=h$.
\end{description}
\subsection{Probability}
\label{sec:probability1}

I hope that the limit in the following definition exists
\begin{equation}
  \newcommand\Nin[2]{N\left(#1,#2, n, n, \frac{\delta}{n}\right)}
  \label{eq:p1}
  P(G|f,h,\delta) \equiv \lim_{n\rightarrow \infty} \frac{
    \sum_{j:G\cap G_{j,n} \neq \emptyset} \Nin{i(f)}{j}\Nin{j}{i(h)}}
  {\sum_j \Nin{i(f)}{j}\Nin{j}{i(h)}}.
\end{equation}
Note that to calculate probability conditioned on only one side rather
than two, one must use an eigenvector of $A$.

\section{Path Volumes}
\label{sec:operator}

In Section~\ref{sec:counting} I divided the unit interval into $n$
segments.  In this section, I replace sums over those intervals with
integrals.  First I replace the adjacency matrix $A$ of
Section~\ref{sec:counting} with the integral operator $\Aop_\tau$
defined in term of the kernel function $\Aindicate_\tau$ that has the
value one if its arguments are an allowed pair
$\left( g(0), g(\tau) \right)$ and zero otherwise, viz:
\begin{subequations}
  \begin{align}
    \label{eq:indicate}
    \Aindicate(u,v) &\equiv
    \begin{cases}
      1 & 0 \leq u \leq 1,~ 0 \leq v \leq 1,~ v \leq u + \tau\\
      0 & \text{otherwise}
    \end{cases}\\
    \label{eq:operateright}
    \left( \Aop_\tau  \function \right)(g) &\equiv \int
            \Aindicate(g,h) \function(h) dh \\
    \label{eq:operateleft}
    \left( \function \Aop_\tau \right)(g) &\equiv \int
           \function(h) \Aindicate(h,g) dh \\
    &= \int_{\max(g-\tau,0)}^1 \function(h) dh.
  \end{align}
\end{subequations}

Consider integer time indices $T$ and introduce the notation
$g_\tau(T)= g(T*\tau)$.  The analog of the number paths for the set of
paths that go from $g_\tau(0)=f$ to $g_\tau(T+1)$ is a \emph{volume}
in $\REAL^T$.  For example if $g_\tau(0) = f$ and $g_\tau(2) = h$ the
volume is the length of the set of allowed values of $g_\tau(1)$,
namely
\begin{align*}
  L &= (f+\tau) - (h-\tau) \\
  &= f-h+2\tau \\
  &= \delta(f) \Aop^2 \delta(h).
\end{align*}
Given a time step $\tau$ the following function is a conditional
probability density for $g_\tau(0)=g$ given $g_\tau(-T)=f$ and
$g_\tau(T)=h$:
\begin{equation*}
  p_\tau(g|f,h,T) = \frac{\delta(f) \Aop^T \delta(g) \Aop^T \delta(h)}
  {\delta(f) \Aop^{2T} \delta(h)}. 
\end{equation*}
As with \eqref{eq:p1} I hope that the limit in
\begin{equation}
  \label{eq:p2}
  p(g|f,h,\delta) \equiv \lim_{n\rightarrow\infty}
  p_{\frac{\delta}{n}} (g|f,h,n)
\end{equation}
exists.

\section{Old Eigenfunction Work}

I use the following notation for an eigenvalue of $\Aop_\tau$ and
corresponding \emph{left} and \emph{right} eigenfunctions:
\begin{align*}
 \left( \leftfunction_\lambda \Aop_\tau \right)(g) &= \lambda
 \leftfunction_\lambda (g)\\
 \left(\Aop_\tau  \rightfunction_\lambda \right)(g) &= \lambda
 \rightfunction_\lambda (g).
\end{align*}
A symmetry argument yields
\begin{equation}
  \label{eq:symmetry}
  \rightfunction_\lambda(g) \propto \leftfunction_\lambda(1-g).
\end{equation}
In the following, since I am only interested in the largest
eigenvalue and the corresponding eigenfunctions\footnote{Called
  the Peron Frobenius eigenvalue and eigenvectors}, I either don't use
subscripts or use them to denote $\tau=s\Delta t$.

Starting from this definition of an eigenfunction
\begin{equation}
  \label{eq:1}
  \lambda_\tau \leftfunction_\tau(g) = \left( \leftfunction_\tau
    \Aop_\tau \right) (g) = \int_0^1 \leftfunction_\tau(h)
  \Aindicate_\tau(h,g) dh,  
\end{equation}
I derive a procedure for constructing eigenfunctions.  Note that
$\Aindicate_\tau(h,g) = 1$ for all $h$ if $g \leq \tau$.
Consequently, the integral in \eqref{eq:1} has the same value for any
$g$ less than $\tau$, ie,
\begin{equation*}
  \int_0^1 \leftfunction_\tau(h) dh \equiv a ~\forall g: g \leq \tau.
\end{equation*}
I can choose to scale the eigenfunction so that $a=\lambda_\tau$ and
\begin{equation*}
  \leftfunction_\tau  (g) = 1 ~\forall g: g \leq \tau.
\end{equation*}
For larger values of $g$ the integrand in \eqref{eq:1} is one for
$g-\tau \leq h < 1$ and
\begin{align}
  \lambda_\tau \leftfunction_\tau(g) &= \int_{g-\tau}^1
      \leftfunction_\tau(h) dh ~~\forall g > \tau \nonumber \\
  &= \int_0^1 \leftfunction_\tau(h) dh - \int_0^{g-\tau}
    \leftfunction_\tau(h) dh \nonumber \\
  &= \lambda_\tau - \int_0^{g-\tau} \leftfunction_\tau(h) dh \nonumber
  \\
  \label{eq:delay_ode}
  \frac{\partial}{\partial g} \leftfunction_\tau(g) &=
            \frac{-1}{\lambda_\tau} \leftfunction_\tau(g-\tau)
             ~~\forall g > \tau .  
\end{align}
Note that for $\tau = 0$, I get the following pair of equations that
has no solutions:
\begin{align*}
  \frac{\partial}{\partial g} \leftfunction_0(g) &=
       \frac{-1}{\lambda_0} \leftfunction_0(g) \\
  \lambda_0 &= \int_0^1 \leftfunction_0(h) dh.
\end{align*}

If $\tau = 1$ then the image of every function $\function$ is a
multiple of the constant function $\leftfunction_\tau(g) = 1$ and the
largest eigenvalue is $1$.  If $\tau = \frac{1}{2}$, then
\begin{equation*}
  \leftfunction_\tau(g) =
  \begin{cases}
    1 & g \leq \frac{1}{2} \\
    1 + \frac{1}{\lambda}\left( \frac{1}{2} - g \right) & g > \frac{1}{2}
  \end{cases},
\end{equation*}
and
\begin{align*}
  \lambda_\tau \leftfunction_\tau (g=0) &= \left( \leftfunction_\tau
                                          \Aop_\tau \right) (g=0) \\ 
  \lambda_\tau &= \left( \int_0^{\frac{1}{2}} dh + \int_{\frac{1}{2}}^1
                1 + \frac{1}{\lambda_\tau}\left( \frac{1}{2} - h \right)
                dh \right) \\
  &= 1 - \frac{1}{8 \lambda_\tau} \\
              &= \frac{1}{2} + \frac{1}{\sqrt{8}}.
\end{align*}

\subsection{Recursion Formula}
\label{sec:recursion}

If $\tau = \frac{1}{n}$, then $\leftfunction_\tau$ is piecewise
polynomial, and using $k$ as the index of the segments starting with
$k=0$, the highest power of $g$ in the $k\th$ segment is $k$.  By
defining $p_k$, a version of the $k\th$ segment rescaled so that it
starts at one, I derive a simple recursion formula for
$\leftfunction_\tau$.  With the definition
\begin{equation}
  \label{eq:pk_def}
  p_k(x) \equiv
  \begin{cases}
    \frac{\leftfunction_\tau(x+k*\tau)} {\leftfunction_\tau(k*\tau)} &
    0 \leq x \leq \tau \\
    \text{undefined} & \text{otherwise}
  \end{cases},
\end{equation}
I find
\begin{align*}
  p_0(x) &= 1 \\
  p_{k+1}(x) &= 1 - \frac{1}{\lambda} \int_0^{x} p_k(s) ds \\
  p_1(x) &= 1 - \frac{x}{\lambda} \\
  p_2(x) &= 1 - \frac{x}{\lambda} + \frac{x^2}{2\lambda^2} \\
  p_k(x) &= \sum_{n=0}^k \frac{1}{n!} \left( -\frac{x}{\lambda}
           \right)^n,
\end{align*}
and in the interval $k\tau \leq g \leq (k+1)\tau$ the eigenfunction is
\begin{equation}
  \label{eq:eigenfunction}
  \leftfunction_\tau(g) = p_k(g-k\tau) \prod_{n=0}^k p_n(\tau).
\end{equation}

\subsection{Approximate Limits}
\label{sec:limits}

Here I seek the behavior of $\lambda_\tau$ and $\leftfunction_\tau$ as
$\tau \rightarrow 0$.  Based on \eqref{eq:delay_ode}, I assume that
for small $\tau$ the eigenfunction has the following approximate form
\begin{equation}
  \label{eq:limit_function}
  \leftfunction(g) \approx
  \begin{cases}
    1 & g \leq \tau \\
    e^{- \frac{g - \tau}{\lambda}} & g > \tau .
  \end{cases}
\end{equation}
From \eqref{eq:1} in the region $g<\tau$ I write
\begin{align}
  \lambda &\approx \tau + e^{\frac{\tau}{\lambda}} \int_\tau^1
  e^{\frac{-h}{\lambda}} dh \nonumber  \\
  &= \tau - \lambda e^{\frac{\tau}{\lambda}} \left[
    e^{\frac{-h}{\lambda}} \right]_\tau^1\nonumber  \\
  &= \tau + \lambda \left( 1 - e^{\frac{\tau -1 }{\lambda}} \right)
    \nonumber \\
  \label{eq:limit_value}
  \tau & \approx \lambda e^{\frac{\tau - 1}{\lambda}}.
\end{align}
I let $\tilde \lambda(\tau)$ denote the value of $\lambda$ that solves
\eqref{eq:limit_value} as an equality.

Figures~\ref{fig:eigenfunctions} and~\ref{fig:eigenvalues} illustrate
$\leftfunction_{\frac{1}{n}}$ and $\lambda_{\frac{1}{n}}$ respectively
for a few values of $n$.\marginpar{Explain algorithms}

\begin{figure*}
  \centering
  \resizebox{0.75\textwidth}{!}{\includegraphics{eigenfunctions.pdf}}
  \caption{Eigenfunctions of $\Aop_\tau$ with $\tau=\frac{1}{n}$ for
    several values of $n$.  Solid lines are from sympy calculations
    and the dotted lines are from numerical power method in which the
    interval is divided into 3,000 pieces.}
  \label{fig:eigenfunctions}
\end{figure*}

\begin{figure*}
  \centering
  %\framebox[0.75\textwidth]{Plot here}
  \resizebox{0.75\textwidth}{!}{\includegraphics{eigenvalues.pdf}}
  \caption{A plot that illustrates dependence of $\lambda_{\tau}$ on
    $\tau$.}
  \label{fig:eigenvalues}
\end{figure*}

\subsection{Probabilities for Discrete Time}
\label{sec:discrete_t}

I begin by analyzing a process with a countable number of values for
the independent variable $t$ with uniform spacing $\tau$.  In this
analysis, I let $\lambda$ denote the largest eigenvalue.  By the Peron
Frobenius Theorem, the corresponding eigenfunctions can be chosen to
be non-negative with normalization that give the stationary
distribution for that process as
\begin{equation}
  \label{eq:stationary}
  \stationary(g) = \leftfunction(g) \rightfunction(g).
\end{equation}
The normalization condition is
\begin{equation*}
  \int \stationary(g) dg = 1.
\end{equation*}
The probability density at a particular sequence of $n+1$ samples from
$g_0$ to $g_n$ is
\begin{equation}
  \label{eq:joint}
  \density(g_0, g_1, \ldots , g_n) = \leftfunction(g_0) \prod_{k=1}^n
  \Aindicate(g_{k-1},g_k) \rightfunction(g_n).
\end{equation}
Thus the conditional density at $g_1$ given $g_0$ after $1$ step is
\begin{align}
  \density(g_1 | g_0) &= \frac{\leftfunction(g_0)
  \Aindicate(g_0,g_1) \rightfunction(g_1)}{ \int \leftfunction(g_0)
  \Aindicate(g_0,h) \rightfunction(h) dh} \nonumber \\
  \label{eq:one_step}
  &= \frac{\Aindicate(g_0,g_1)
    \rightfunction(g_1)}{\lambda \rightfunction(g_0)}.
\end{align}
Similarly, for two steps I find
\begin{align}
  \density(g_2 | g_0) &= \frac{\leftfunction(g_0) \left( \int
      \Aindicate(g_0,h_1) \Aindicate(h_1,g_2) dh_1 \right)
      \rightfunction(g_2) } 
      { \leftfunction(g_0) \int
      \Aindicate(g_0,h_1) \Aindicate(h_1,h_2) \rightfunction(h_2) dh_1
      dh_2} \nonumber \\
  &= \frac{ \left( \int
      \Aindicate(g_0,h_1) \Aindicate(h_1,g_2) dh_1 \right)
      \rightfunction(g_2) } 
      { \lambda^2  \rightfunction(g_0)} \nonumber \\
  \label{eq:two_step}
  &\equiv \frac{\delta(g_0) \Aop \Aindicate(\cdot,g_2)
    \rightfunction(g_2)}{\lambda^2 \rightfunction(g_0)},
\end{align}
and for $n$ steps
\begin{equation}
  \label{eq:n_step}
  \density(g_n | g_0) = \frac{\delta(g_0) \Aop^{n-1} \Aindicate(\cdot,g_n)
    \rightfunction(g_n)}{\lambda^n \rightfunction(g_0)}.  
\end{equation}
Note that
\begin{equation*}
  \delta(g_0) \Aop \Aindicate(\cdot,g_2) \equiv \int
  \Aindicate(g_0,h_1) \Aindicate(h_1,g_2) dh_1
\end{equation*}
is the length of the interval of $g_1$ values that are possible given
the values $g_0$ and $g_2$, and that the generalization
$\delta(g_0) \Aop^{n-1} \Aindicate(\cdot,g_n)$ is the volume in
$\REAL^{n-1}$ of tuples $(g_1,g_2,\ldots,g_{n-1})$ that are possible
given $g_0$ and $g_n$.

\subsection{Continuous Time}
\label{sec:continuous_t}

The principal task of this paper is to determine if a continuous time
limit of \eqref{eq:n_step} exits.  In particular, does the right hand
side of \eqref{eq:conditional} converge?
\begin{align}
  \label{eq:conditional}
  \density(g_t|g_0) = &\lim_{n \rightarrow \infty}  \frac{\delta(g_0)
    \left( \Aop_\tau \right)^{n-1} \Aindicate_\tau(\cdot,g_t)
    \rightfunction_\tau(g_t)} { \left( \lambda_\tau \right)^n
    \rightfunction_\tau(g_0)} \\
  & \text{ with } \tau=\frac{t}{n} \nonumber
\end{align}
Since Eqns.~\eqref{eq:limit_function} and \eqref{eq:limit_value}
provide\marginpar{Wrong}
$\lim_{\tau \rightarrow 0} \rightfunction_\tau(g) = e^{g-\frac{1}{2}}$
and $\lim_{\tau \rightarrow 0} \lambda_\tau = 1 - \frac{1}{e}$, all
that remains is to show that the limit in \eqref{eq:paths} exists.
\begin{align}
  \label{eq:paths}
  T_{t,g_0} & \equiv \lim_{n \rightarrow \infty} T_{t,g_0,n}, \text{
  where}\\
  T_{t,g_0,n} & \equiv \delta(g_0) \left( \frac{\Aop_{\frac{t}{n}}}
                {\lambda_{\frac{t}{n}}} \right)^n  \nonumber
\end{align}

For large $t$ the limit in \eqref{eq:paths} approaches a power method for
finding $\leftfunction$, and
\begin{equation*}
  \lim_{t \rightarrow \infty} T_{t,g_0} \propto \leftfunction.
\end{equation*}
However, if $t$ is smaller than $1 - g_0$ then the monotonic
constraint implies that
\begin{equation*}
  T_{t,g_0,n} = 0 ~\forall g,n: g > g_0 + t \text{ and } n > 0.
\end{equation*}
To simplify numerical calculations, I define
\begin{equation*}
  \tilde T_{t,g_0,n} \equiv \delta(g_0) \left( \frac{\Aop_{\frac{t}{n}}}
                {\lambda_0} \right)^n.
\end{equation*}
Numerical calculations of $\tilde T_{t=\frac{1}{4},g_0=\frac{1}{2},n}$
for several values of $n$ appear in Figure~\ref{fig:paths} and that
numerical evidence supports the conjecture that \eqref{eq:paths}
converges.
\begin{figure*}
  \centering
  \framebox[0.75\textwidth]{Plots here}
  %\resizebox{0.75\textwidth}{!}{\includegraphics{paths.pdf}}
  \caption{These plots of $\tilde T_{t=\frac{1}{4},g_0=\frac{1}{2},n}$
    for several values of $n$, suggest that $\lim_{n \rightarrow
      \infty}$ exists.}
  \label{fig:paths}
\end{figure*}


\newpage
\section{A simple example}
\label{sec:example}

Consider the adjacency matrix
\begin{equation}
  \label{eq:A}
  A = \begin{bmatrix} 1 & 1 \\ 1 & 0 \end{bmatrix},
\end{equation}
the schematic of the corresponding finite state Markov process in
Figure~\ref{fig:mt2}, and its probability matrix
\begin{equation}
  \label{eq:T}
  \T = \begin{bmatrix} a & b \\ c & 0 \end{bmatrix}.
\end{equation}

\begin{figure*}
  \centering
  \resizebox{0.4\textwidth}{!}{\input{mt2.pdf_t} }
  \caption{A schematic of the finite state Markov process specified by
  Equation~\eqref{eq:T}.}
  \label{fig:mt2}
\end{figure*}

Given values for $a$, $b$, and $c$, one can solve
\begin{equation}
  \label{eq:stationary_2}
  \mu = \mu \T
\end{equation}
for the stationary distribution $\mu$ and from there find the entropy
rate $h$ by
\begin{equation}
  \label{eq:rate}
  h = \sum_i \mu_i H(J|i)
\end{equation}
where
\begin{equation*}
  H(J|i) \equiv -\sum_j \T_{i,j} \log (\T_{i,j})
\end{equation*}
is the familiar $P\log(P)$ formula.

\subsection{Parameters of $\T$ that maximize $h$}
\label{sec:max}

For the maximum entropy values, given a trajectory length $N$, all
trajectories must have almost the same probability.  Since all
trajectories are composed of loops in a cycle basis, the following
equation must hold
\begin{equation}
  \label{eq:excycle}
  a^2 = b\cdot c.
\end{equation}
That combined with the normalization constraints for the two nodes
\begin{align*}
  a + b &= 1 \\
  c &= 1
\end{align*}
is sufficient to specify the solution
\begin{align*}
  a &= \frac{2}{1+\sqrt{5}} \approx          0.61803\\
  b &= \frac{\sqrt{5}-1}{1+\sqrt{5}} \approx 0.38197\\
  c &= 1.
\end{align*}
Solving \eqref{eq:stationary_2} for that specification of $\T$ yields
\begin{align*}
  \mu &=  \begin{bmatrix} \frac{1+\sqrt{5}}{2\sqrt{5}}, &
    \frac{\sqrt{5}-1}{2\sqrt{5}} \end{bmatrix} \\
  h &= \log\left(\frac{1+\sqrt{5}}{2}\right).
\end{align*}


\subsection{Topological entropy $\htop$}
\label{sec:htop}

The topological entropy of a directed graph specified by an adjacency
matrix $A$ is the rate at which the number of trajectories grows with
length.  If $n(t)\equiv\begin{bmatrix} n_1(t),&n_2(t)\end{bmatrix}$
denotes the number of trajectories of length $t$ that end in the two
states of Figure \ref{fig:mt2}, then
\begin{align*}
  n(1) &= \begin{bmatrix} 1,&1\end{bmatrix} \text{ and} \\
  n(t+1) &= n(t) A.
\end{align*}
Thus
\begin{align*}
  n(t) &= \begin{bmatrix} 1,&1\end{bmatrix} A^t \\
  \text{and }\lim_{t\rightarrow \infty} \frac{1}{t} \log n_1(t) &=
  \lambda
\end{align*}
where $\lambda$ is the largest eigenvalue of $A$.

Solving the eigenvalue problem yields
\begin{equation*}
  \htop = \log\left(\frac{1+\sqrt{5}}{2}\right).
\end{equation*}

The topological entropy provides one more equation of constraint on
the equations that the values in $\T$ must satisfy for maximizing $h$.
Rather than the single equation \eqref{eq:excycle}, the pair
\begin{equation*}
  - \htop = 2\log(a) = \log(b) + \log(c)
\end{equation*}
must hold.

\section{A Power Iteration Approach}
\label{sec:algorithm}

I will use the solution to this problem to characterize the
distribution of long trajectories in a system defined by an adjacency
matrix $A$.  I want the distribution to be uniform over all allowed
trajectories.

Given a list of allowed trajectories of length $2T$, I could
approximate the value of $\T_{i,j}$ by
\begin{equation*}
  \hat \T_{i,j} = \frac{_{2T}n_{T,T+1}(i,j)}{_{2T}n_{T}(i)},
\end{equation*}
where $_{2T}n_{T}(i)$ denotes the number of trajectories of length
$2T$ for which $i$ is the value at position $T$ and similarly
$_{2T}n_{T,T+1}(i,j)$ denotes the number of trajectories that have
values $i$ and $j$ at positions $T$ and $T+1$ respectively.  While the
exponential growth of the number of trajectories suggests that
implementing the estimate for large $T$ is not feasible, power
iterations for left and right eigenvectors of $A$ make it easy.

The number of allowed sequences of length say six that exactly matches
a given sequence is either zero or one and can be written as
\begin{equation*}
  _6n_{1,2,3,4,5,6}(a,b,c,d,e,f) = A_{a,b}A_{b,c}A_{c,d}A_{d,e}A_{e,f}.
\end{equation*}
The number of sequences that are only required to match at positions
three and four is
\begin{equation*}
 _6n_{3,4}(c,d) =  \sum_{a,b,e,f} {_6n}_{1,2,3,4,5,6}(a,b,c,d,e,f) =
 \sum_{a,f} \left(A^2\right)_{a,c} A_{c,d} \left(A^2\right)_{d,f}.
\end{equation*}
Similarly
\begin{equation*}
 _{202}n_{101,102}(c,d) = \sum_{a,f} \left(A^{100}\right)_{a,c} A_{c,d} \left(A^{100}\right)_{d,f}.
\end{equation*}

I define (and calculate) $R(t)$ and $L(t)$ recursively with the
following power iteration scheme
\begin{align*}
  L(1) &= \begin{bmatrix} 1,&1,&\cdots,&1,&1\end{bmatrix} \\
  N_L(t) &= \left| L(t) \right| \\
  L(t+1) &= \frac{L(t) A}{N_L(t)} \\
  R(1) &= L^{\text{T}}(1) \\
  N_R(t) &= \left| R(t) \right| \\
  R(t+1) &= \frac{A R(t)}{N_R(t)}.
\end{align*}
Note that as $t$ increases, $R(t)$ and $L(t)$ converge quickly to the
right and left eigenvectors respectively of $A$ that correspond to the
largest eigenvalue and I can calculate
\begin{equation}
  \label{eq:N}
   _{2(T+1)}n_{T+1,T+2}(c,d) \propto \left(L(T)\right)_c  A_{c,d}
   \left(R(T)\right)_d
\end{equation}
pretty easily.  With results from \eqref{eq:N} for a large enough $T$
to ensure convergence, one can calculate estimates $\hat \mu$ and
$\hat \T$ of the stationary distribution and transition probabilities
respectively as follows.
\begin{align*}
  \tilde L &= L(T) \\
  \tilde R &= R(T) \\
  \tilde P_{i,j} &= \tilde L_i A_{i,j} \tilde R_j \\
  \tilde m_i &= \sum_j \tilde P_{i,j} \\
  \hat \mu &= \frac{\tilde m}{\sum_i \tilde m_i} \\
  \hat \T_{i,j} &= \frac{\tilde P_{i,j}}{\tilde m_i}
\end{align*}
Maxentropic Markov chains appears in:\\
IEEE Transactions on Information Theory, \\
Date of Publication: Jul 1984\\
Author(s): Justesen, J.\\
Hoholdt, T.\\
Volume: 30 , Issue: 4\\
Page(s): 665 - 667

\section{Composition}
\label{sec:composition}

\begin{align*}
  \left( P(0\mapsto 1) \right)_{i,j} &= \frac{L_i A_{i,j} R_j}
    {\lambda z \frac{\sum_k L_i A_{i,k} R_k}{\lambda z}} \\
  &= \frac{A_{i,j} R_j}{\lambda R_i} \\
  \left( P(0\mapsto 2) \right)_{i,j} &=
     \sum_k \frac{A_{i,k} R_k}{\lambda R_i}
            \frac{A_{k,j} R_j}{\lambda R_k} \\
  &= \frac{(A^2)_{i,j} R_j}{\lambda^2 R_i}                                       
\end{align*}
If $B$ is the adjacency matrix for twice the step size of $A$,
sufficient conditions for consistency are
\begin{align*}
  B R &= \gamma R \\
  \frac{B_{i,j} R_j}{\gamma R_i} &= \frac{(A^2)_{i,j} R_j}{\lambda^2
                                   R_i} \text{or}\\
  (A^2)_{i,j} &= \frac{\lambda^2}{\gamma} B_{i,j}.
\end{align*}
The conditions are not plausible because if it is possible to get from
$i$ to $j$ in two steps, ie, $B_{i,j}=1$, the number of ways to get
from $i$ to $j$ in two steps will vary depending on the values of $i$
and $j$.

\section{Repetition}
\label{sec:repetition}

Here I've typed up hand written alternative notation.  For the one
step joint probability
\begin{align*}
  P_{j,k} &= \lim_{n\rightarrow \infty} \frac {\sum_{i,l} \left(A^n
            \right)_{i,j} A_{j,k}  \left(A^n \right)_{i,j} }{N_n}
            \text{ where the normalization } N_n \text{ makes} \\
  \sum_{j,k}P_{j,k} &= 1   \\
  P_{j,k} &= L_j A_{j,k} R_k \text{ where } L,R \text{ are the left and right
    eigenvectors of } A \\
  P_{k|j} &= \frac{L_j A_{j,k} R_k}{\sum_k L_j A_{j,k} R_k} \\
  &= \frac{L_j A_{j,k} R_k}{\lambda L_j R_j} \\
  &= \frac{A_{j,k} R_k}{\lambda R_j}
\end{align*}
For $n$ steps rather than one, the same calculations yield
\begin{align*}
  P_{k|j}(x(n)|x(0)) &= \frac{\left( A \right)^n_{j,k} R_k}{\lambda^n
                       R_j} \\
  &\propto \left( \delta_j \cdot A^n \right)_k R_k
\end{align*}


\end{document}

%%%---------------
%%% Local Variables:
%%% eval: (TeX-PDF-mode)
%%% eval: (setq ispell-personal-dictionary "./localdict")
%%% End:
