\documentclass[twocolumn]{article}
\usepackage{amsmath,amsfonts,afterpage}
\usepackage{showlabels}
\usepackage[pdftex]{graphicx,color}
\newcommand{\normal}[2]{{\cal N}(#1,#2)} \newcommand{\La}{{\cal L}}
\newcommand{\nomf}{\tilde f} \newcommand{\COST}{\cal C}
\newcommand{\LL}{{\cal L}} \newcommand{\Prob}{\text{Prob}}
\newcommand{\field}[1]{\mathbb{#1}}
\newcommand\REAL{\field{R}}
\newcommand\Z{\field{Z}}
\newcommand\Polytope[1]{\field{P}_{#1}}
\newcommand\PolytopeN{\Polytope{N}}
\newcommand\PolytopeInf{\Polytope{\infty}}
\newcommand{\EV}[2]{\field{E}_{#1}\left[#2\right]}
\newcommand{\partialfixed}[3]{\left. \frac{\partial #1}{\partial
      #2}\right|_#3}
\newcommand{\dhot}{=_{\text{dhot}}}
\newcommand{\argmin}{\operatorname*{argmin}}
\newcommand{\argmax}{\operatorname*{argmax}}
\newcommand{\set}[1]{{\cal #1}}
\newcommand\inner[2]{\left<#1,#2\right>}
\newcommand\norm[1]{\left|#1\right|}
\newcommand\bv{\mathbf{v}}
\newcommand\bw{\mathbf{w}}
\newcommand\dum{\xi}
\newcommand\Ddum{d\dum}
\newcommand\lambdaPF{\lambda_{\rm{PF}}} % Peron Frobenius eigenvalue
\newcommand\rhoPF{\rho_{\rm{PF}}} % Peron Frobenius eigenvector

\title{Quadratic Approximation to First Order Integral Operator}

\author{Andrew M.\ Fraser}
\begin{document}
\maketitle
\begin{abstract}
  Derived via \emph{notes} from \emph{Propagating Uncertainty About
    Gas EOS to Performance Bounds for an Ideal Gun} LA-UR-12-22731.
  I describe simulations and calculations on an approximation to an
  integral operator.
\end{abstract}
\section{Introduction}
  \label{sec:introduction}

Here I consider a first order Markov process on 2-d states consisting
of function values $g(y)$ and derivatives $h(y) \equiv \left. \frac{d
    g }{d t} \right|_{t=y}$.  I consider sequential samples at $y_0$
and $y_1$, and I suppose that the upper bound is $u$ and the lower
bound is $l=-u$.

In $(f,x)$ coordinates, I describe the nominal function, $\tilde f$ by
\begin{align*}
  x_0 &= 1\\
  f_0 &= f(x_0) \\
  \tilde f(x) &= \frac{\tilde f_0}{x^3}.
\end{align*}
The coordinates $g$ and $y$ are functions of coordinates $f$ and $x$ with
\begin{align*}
  x(y) &= e^y \\
  f(g) &= \frac{\tilde f_0 e^{g}}{x^3} \\
  g(f) &= \log(f) + 3y - \log(\tilde f_0).
\end{align*}
The straight line in $(g,y)$, $g(y) = 0$, corresponds to
\begin{equation*}
  \tilde f(x) = \frac{\tilde f_0}{x^3}.
\end{equation*}

In the $(f,x)$ coordinates a line that is tangent to $\tilde f(x)$ at
$x=1$ satisfies
\begin{equation*}
  f(x) = f_0(4-3x),
\end{equation*}
which in the $(g,y)$ coordinates is
\begin{equation*}
  g(y) = \log \left( 4 - 3e^y \right) + 3y.
\end{equation*}
At $y=0$, that $g$ has the following derivatives
\begin{align*}
  g(0) & = g_0 \\
  g'(y) &= 3 - \frac{3e^y}{4-3e^y} \\
  g'(0) &= 0 \\
  g'' &= -\frac{\frac{4}{3}e^y}{\left( e^y - \frac{4}{3} \right)^2}\\
  g''(0) &= -12 \\
  g''' &= -\frac{4\left( \frac{4}{3} + e^y \right) e^y}{
    3\left( \frac{4}{3} - e^y \right) ^3} \\
  g'''(0) &= -84.
\end{align*}
Thus the second order Taylor series approximation to the function in
$(g,y)$ that corresponds to a tangent to $\frac{\tilde f_0
  e^{g_0}}{x^3}$ at $x_0 = 1$ is
\begin{equation}
  \label{eq:tangent}
  g(y) = a - 6(y-b)^2.
\end{equation}
I will use the Taylor series approximation \eqref{eq:tangent} to
tangents in $(f,x)$ from now on.  Figure~\ref{fig:taylor} illustrates
that approximation.

\begin{figure}
  \centering
    \resizebox{\columnwidth}{!}{\includegraphics{taylor.pdf}}  
    \caption{Second order Taylor series approximation to $g(y)$
      compared to actual $g(y)$ when $f = e^g = \alpha - \beta x$,
      $y=\log(x)$, and $f$ is tangent to $\frac{1}{x^3}$ at $x=1$.
      The indicated bounds, $\pm 0.0005 = 5\times10^{-4}$ correspond
      to a factor of $1.0005$ which is one tenth the value quoted by
      Shaw.}
  \label{fig:taylor}
\end{figure}
%\end{document}
Now I describe quadratic approximations to constraints on the values
of $g_1 \equiv g(y_1)$ and $h_1 \equiv h(y_1)$ that are determined by
$g_0 \equiv g(y_0)$ and $h_0 \equiv h(y_0)$.  I use $U_g$, $L_g$,
$U_h$, and $L_h$ to denote (upper/lower) bounds on ($g_1/h_1$)
respectively.  A trajectory $g$ that goes through $g_0$ at $y_0$ must
lie between the two lines $U_g(g_0,y_0,y)$ and $\bar U_g(g_0,y_0,y)$
that in $(f,x)$ coordinates go through $f_0$ and are tangent to the
upper bound $u$ to the right and left of $y_0$ respectively.  If $h_0$
is the derivative of $g$ at $y_0$, the value of $g$ at $y_1$, ie,
$g_1$ must lie above $L_g(g_0,h_0,y_0, y_1)$, the image in $(g,y)$ of
the tangent to $f$ at $x_0$ in $(f,x)$.  Thus
\begin{equation}
  \label{eq:boundsA}
  U_g(g_0,y_0, y_1) \geq g_1 \geq L_g(g_0,h_0,y_0, y_1).
\end{equation}
The following calculations fit quadratic approximations to both $U_g$
and $L_g$.
\newcommand{\Rad}[1]{\sqrt{24\left(u#1\right)}}
\newcommand{\UgQ}{ g_0 + \Delta_y\Rad{-g_0} - 6 \Delta_y^2}
\begin{align}
  U_g(y) &= u - 6(y-b)^2 &&\text{Premise} \nonumber \\
  U_g(y_0) &= g_0 &&\text{Premise} \nonumber \\
  \label{eq:premise}
  U'_g(y_0) &\geq 0 &&\text{Premise} \\
  g_0 &= u - 6(y_0-b)^2 \nonumber \\
  \label{eq:root}
  b &= y_0 \pm \sqrt{\frac{u-g_0}{6}} \\
  \label{eq:dugb}
  \frac{d U_g}{d y} &= -12(y-b) && \text{Positive root in \eqref{eq:root}} \\
  \label{eq:dug}
  \left. \frac{d U_g}{d y} \right|_{y_0} &= 12\sqrt{\frac{u-g_0}{6}}
\end{align}
\begin{equation*}
  U_g(y) = g_0 + (y-y_0)\Rad{-g_0} - 6 (y-y_0)^2
\end{equation*}

Note: If $U_g(y)$ is tangent to $y=u$ for some $y$ between $y_0$ and
$y_1$, then the only upper bound on $g(y_1)$ is $g(y_1)\leq u$.  Also
such a tangency exists iff
\begin{equation*}
  \sqrt{\frac{u-g_0}{6}} \leq \Delta_y.
\end{equation*}
Thus
\begin{equation}
  \label{eq:ug}
  U_g(y_1) =
  \begin{cases}
    u & \sqrt{\frac{u-g_0}{6}} \leq \Delta_y\\
    \UgQ & \text{ otherwise}
  \end{cases}
\end{equation}
Similarly, for the lower bound:
\newcommand{\LgQ}{g_0 + h_0\Delta_y - 6\Delta_y^2}
\begin{align}
  L_g(y) &= a - 6(y-b)^2 &&\text{Premise} \nonumber \\
  L'_g(y) &= -12(y-b) \nonumber \\
  g_0 &= a - 6(y_0-b)^2 &&\text{Premise} \nonumber \\
  h_0 &= -12(y_0-b) &&\text{Premise} \nonumber \\
  b &= y_0 + \frac{h_0}{12} \nonumber \\
  a &= g_0 + \frac{h^2_0}{24} \\
  \label{eq:lg}
  L_g(y_1) &= \max \left( -u,~\LgQ \right)
\end{align}
Substituting these values into \eqref{eq:boundsA} yields\footnote{The
  upper bound in Eqn.\ \eqref{eq:UL_g} is simply $u$ if $g_0 \geq u -
  6 \Delta_y^2$, and the lower bound in Eqn.\ \eqref{eq:DG_DY} is only
  valid if $\sqrt{\frac{u-g_0}{6}} \geq \Delta_y$.}
\begin{align}
  \label{eq:UL_g}
  \UgQ &\geq g_1 \geq U_g(y_1) \\
  \label{eq:DG_DY}
  \Rad{-g_0} - 6\Delta_y &\geq \frac{\Delta_g}{\Delta_y} \geq h_0
  - 6 \Delta_y
\end{align}
Figure~\ref{fig:boundsA} illustrates the bounds
\eqref{eq:boundsA}.  In Fig.~\ref{fig:boundsB}, I have simply reduced
the extent in $y$.

\begin{figure}
  \centering
    \resizebox{\columnwidth}{!}{\includegraphics{bounds_04.pdf}}  
  \caption{Bounds on $g_1$ given $g_0$ and $h_0$}
  \label{fig:boundsA}
\end{figure}
\begin{figure}
  \centering
    \resizebox{\columnwidth}{!}{\includegraphics{bounds_005.pdf}}  
  \caption{Bounds on $g_1$ given $g_0$ and $h_0$}
  \label{fig:boundsB}
\end{figure}

Given $g_0$ and a $g_1$ that satisfies the constraints
\eqref{eq:boundsA}, the constraints on $h_1$ are that it must be less
than the slope $\left. \frac{d U_g(g_1,y_1,y)}{dy} \right|_{y=y_1}$
and it must be greater than the slope at $y_1$ of the image in $(g,y)$
of the line in $(f,x)$ that connects the images of $(g_0,y_0)$ and
$(g_1,y_1)$.  The following calculation finds the quadratic
approximation for that curve:
\begin{align*}
  g(y) &= a -6(b-y)^2 \\
  g_0 &= a -6(b-y_0)^2 & g_1 &= a -6(b-y_1)^2 \\
  a &= g_0 + 6(b-y_0)^2 =  g_1 + 6(b-y_1)^2 \\
  g_0 -g_1 &= 12b(y_0-y_1) + 6(y_1-y_0)(y_1+y_0) \\
  b &= \frac{g_1-g_0 + 6 (y_1^2-y_0^2)}{12(y_1-y_0)} = \frac{g_1 -
    g_0}{12 (y_1 - y_0)} + \frac{y_1 + y_0}{2}\\
  a &= \frac{\Delta_g^2}{24 \Delta y ^2} + \frac{g_1 + g_2}{2} +
  \frac{ 3 \Delta_y^2}{2}.
\end{align*}
At $y_1$, the derivative of $a-6(y-b)^2$ is
\newcommand{\LhQ}{\frac{\Delta_g}{\Delta_y} - 6 \Delta_y}
\begin{align*}
  L_h(g_1, g_0, y_0, y_1) &= -12(y_1-b) \\
  &= \frac{g_1 - g_0 -6(y_1 - y_0)^2} {y_1-y_0} \\
  & \equiv \LhQ
\end{align*}
Following \eqref{eq:dug}, the upper bound is
\newcommand{\Uh}{\Rad{-g_1}}
\begin{equation*}
  U_h(g_1) = \Uh,
\end{equation*}
and\footnote{The \emph{maximum} function and the term $-\Uh$ treat the
case of $U_g(y)$ being tangent to $u$ between $y_0$ and $y_1$.}
\begin{equation}
  \label{eq:UL_h}
  \Uh \geq h_1 \geq \text{max}\left( \LhQ,~ -\Uh \right).
\end{equation}

Figure~\ref{fig:boundsC} illustrates those constraints.

\begin{figure}
  \centering
    \resizebox{\columnwidth}{!}{\includegraphics{bounds_dg.pdf}}  
    \caption{Curves $U_h(g_1)$ and $L_h(g_1, g_0)$ whose tangents give
      bounds on $h_1$ given $g_0$ and $g_1$}
  \label{fig:boundsC}
\end{figure}

\section{Scaling and Pie Slices}
\label{sec:scaling}

Eric Mjolsness suggested calculating moments of the images of each
allowed point in $G \times H$.  Before doing the honest work, I
considered the kinds of images that occur and noticed that there is
only one parameter in the problem.  While both $\Delta_y$ and $u$
appear in \eqref{eq:UL_g} and \eqref{eq:UL_h}, in the alternative
coordinates in which $\Delta_y=1$, ie,
\begin{align*}
  \tilde g(g) &= \frac{g}{\Delta_y^2} \\
  \tilde h(h) &= \frac{h}{\Delta_y} \\
  \tilde u(u) &= \frac{u}{\Delta_y^2}
\end{align*}
it is clearer that the behavior near $g=u$ does not depend on
$\Delta_y$, and that the ratio $\frac{u}{\Delta_y^2}$ is the number
that characterizes the problem.

So, without loss of generality, in this section, I assume $\Delta_y=1$
and let $u$ define the problem.  The domain $D$ of the problem is the set
of points in $G\times H$ that satisfy
\begin{subequations}
  \label{eq:domain}
  \begin{align}
    -u &\leq g \leq u \\
    h^2 &\leq 24(u-g)
  \end{align}
\end{subequations}
The image of each point $(g_0, h_0)$ is a \emph{pie slice} defined by
the intersection of a wedge and the domain $D$.  Each wedge has an
apex at $(g_0+h_0-6, h_0-12)$, and from the apex, one edge is vertical
and the other has unit slope $\frac{d\, h}{d\, g} = 1$.  I consider
the following three kinds of pie slices, $P$
\begin{description}
\item[Left end:] For some $(g,h) \in P$, $g=-u$
\item[Right end:] For some $(g,h) \in P$, $g=u$
\item[Interior:] All other cases
\end{description}

The equation
\begin{subequations}
  \label{eq:boundary}
  \begin{align}
  g &= u -\frac{h^2}{24}\\
  g &= =u
  \end{align}
\end{subequations}
are the boundaries of $D$.
\section{Integral Equations}
\label{sec:integral-equations}

I will use the following notation:
\begin{description}
\item[$y$] The independent variable, $y = \log(x)$
\item[$z(y)$] A state consisting of $g(y)$, the value of the function
  and $h(y) \equiv \left. \frac{d g(t)}{d t} \right|_y$, its derivative.
\item[$I_\Delta$] The indicator function:
  \begin{equation}
    \label{eq:indicator}
    I_\Delta(z_0, z_1) =
    \begin{cases}
      1 & \text{if } \left( z(0) = z_0,~z(\Delta)=z_1
      \right) \text{ is allowed} \\
      0 & \text{otherwise}
    \end{cases}
  \end{equation}
  The subscript of $I_\Delta$ is a distance in $y$.  I will drop it
  unless I need it.
\item[$\Phi_\Delta$] The linear operator consisting of integration
  over $I_\Delta$
\item[${\lambdaPF}_\Delta$] The eigenvalue of $\Phi_\Delta$ with the largest
  magnitude.  The Peron-Frobenius theorem promises that
  ${\lambdaPF}_\Delta$ is isolated and positive and that there is a
  corresponding eigenfunction that is strictly positive.
\item[${\rhoPF}_\Delta$] The positive unit norm eigenfunction that
  corresponds to ${\lambdaPF}_\Delta$; defined by
  \begin{equation}
    \label{eq:integralI}
    \int I(z_0, z_1) {\rhoPF}_\Delta(z_1) d z_1 \equiv {\lambdaPF}_\Delta
    {\rhoPF}_\Delta(z_0).
  \end{equation}
  I will drop the subscripts of ${\lambdaPF}_\Delta$ and
  ${\rhoPF}_\Delta$ when I can without creating ambiguity.
\item[$U_g,~L_g,~U_h,~L_h$] Functions that provide integration limits
  that express \eqref{eq:integralI} as
  \begin{equation}
    \label{eq:integralUL_}
    \int_{L_g(g_0,h_0)}^{U_g(g_0)} dg_1
    \int_{L_h(g_1,g_0)}^{U_h(g_1)} dh_1~ \rho(g_1, h_1) = \lambda
    \rho(g_0, h_0)
  \end{equation}
\end{description}

From the quadratic approximations \eqref{eq:UL_g} and \eqref{eq:UL_h}
I write the bounds in \eqref{eq:integralUL_} as follows:
\begin{equation*}
  L_g(g_0, h_0) = \max \left( -u, g_0 + h_0 \Delta_y - 6 \Delta_y^2
  \right)  
\end{equation*}
\begin{align*}
  &U_g(g_0) = \\
  &\begin{cases}
    u & g_0 > u - 6 \Delta_y^2 \\
    g_0 + \Delta_y\Rad{-g_0} - 6 \Delta_y^2 & \text{otherwise}
  \end{cases}
\end{align*}
\begin{align*}
  L_h(g_0, g_1) &= \max \left( \frac{\Delta_g}{\Delta_y} - 6
    \Delta_y, - \Rad{-g_1} \right) \\
  U_h(g_1) &= \Rad{-g_1} ,
\end{align*}
and then
\newcommand{\Ug}{ g_0 + \Delta_y\Rad{-g_0}}
\newcommand{\Lg}{g_0 + h_0\Delta_y}
\newcommand{\Lh}{\frac{\Delta_g}{\Delta_y}}
\begin{align}
  \lambda \rho(z_0) &= \int I(z_0, z_1) \rho(z_1) d
  z_1 \nonumber \\
  \label{eq:integralULq}
    &= \int_{L_g(g_0, h_0)}^{U_g(g_0)} dg_1
    \int_{L_h(g_0,g_1)}^{U_h(g_1)}~ \rho(h_1,g_1) ~dh_1.
\end{align}

\section{Discrete Approximation}
\label{sec:approximate}


\subsection{2-d Illustrations}
\label{sec:ill}

I use $A$ to denote a discrete approximation of the integral operator,
and let $A(g,h)$ denote the set of components in the image of the
single component $(g,h)$.  $A$ maps states $(g,h)$ at $y$ to states
$(g',h')$ at $y+\Delta_y$, and while $A^T$ maps from $y$ to
$y-\Delta_y$, it is not the inverse of $A$.  If
\begin{equation*}
  (g',h') \in A(g,h),
\end{equation*}
then
\begin{equation*}
  (g', -h') \in A^T(g,-h).
\end{equation*}
Thus, if $S$ denotes the interchange of $h$ and $-h$
\begin{align*}
  SS &= I \\
  SA^TS &= A,
\end{align*}
which ought to be worth something.

Figures \ref{fig:Av1}-\ref{fig:Av3} illustrate the action of $A$ and
$A^T$, and a numerically estimated eigenfunction appears in
Fig.~\ref{fig:eigenfunction}.
\begin{figure}
  \centering
    \resizebox{1.0\columnwidth}{!}{\includegraphics{Av1.pdf}}
    \caption{Image under $A^T$ of five points.  For each starting
      point $(g_0,h_0)$, the set of possible image points $\left\{
        (g_1,h_1) \right\}$ is a wedge with a vertex at
      $(g_0+h_0\Delta_y - 6\Delta_y^2,h_0 - 12\Delta_y)$.  The right
      face of each wedge has a slope of $\frac{1}{\Delta_y}$.  For
      points near the right end of the allowed region with $g_0 > u -
      6 \Delta_y^2$, the images have special shapes.}
  \label{fig:Av1}
\end{figure}

\begin{figure}
  \centering
    \resizebox{1.0\columnwidth}{!}{\includegraphics{Av2.pdf}}
    \caption{Image under $A$ of the same five points as used in
      Fig.~\ref{fig:Av1}.  For each starting point $(g_0,h_0)$, the
      set of possible pre-image points $\left\{ (g_{-1},h_{-1})
      \right\}$ is a wedge with a vertex at $(g_0-h_0\Delta_y +
      6\Delta_y^2,h_0 + 12\Delta_y)$.  For any point above the
      diagonal line in the upper left of the plot, the pre-image is
      truncated by the left edge of the allowed region.  That
      truncation explains the small values of the eigenfunction in
      that region.}
  \label{fig:Av2}
\end{figure}

\begin{figure}
  \centering
    \resizebox{1.0\columnwidth}{!}{\includegraphics{Av3.pdf}}
    \caption{Image of points under $A$ with a smaller value of
      $\Delta_y$.  Note that the the region of truncated pre-images is
      smaller than for Fig.~\ref{fig:Av2}.  However, since the point
      just to the right of that region has a pre-image that intersects
      the region, the eigenfunction will be suppressed there in a second
      order fashion.}
  \label{fig:Av3}
\end{figure}

\begin{figure}
  \centering
    \resizebox{1.0\columnwidth}{!}{
      \includegraphics{figs/2000_g_800_h_2_y_vec_log.png}}
    \caption{The $\text{log}_{10}$ of the amplitude of a numerically
      estimated eigenfunction.}
  \label{fig:eigenfunction}
\end{figure}

\onecolumn
\section{Analysis of the Integral Equation}
\label{sec:solve}

Differentiating
\eqref{eq:integralULq} once with respect to $h_0$ eliminates the
integral over $dg_1$, requires the substitution
\begin{equation*}
  g_1 = L_g(g_0,h_0) = \LgQ \equiv L_g,
\end{equation*}
and using
\begin{equation*}
  \frac{\partial L_g(g_0,h_0)}{\partial h_0} = \Delta_y
\end{equation*}
yields
\begin{align*}
  \lambda \frac{\partial}{\partial h_0} \rho(g_0,h_0) &= -
  \frac{\partial L_g(g_0,h_0)}{\partial h_0} \int_{L_h(L_g(g_0,h_0),
    g_0)}^{U_h(L_g(g_0,h_0))} \rho(L_g(g_0,h_0),h_1) d h_1 \\
  & = -\Delta_y \int_{h_0 - 12\Delta_y}
  ^{\sqrt{24(u-g_0-h_0\Delta_y + 6\Delta_y^2)}}~ 
  \rho(g_0+h_0 \Delta_y -6 \Delta_y^2 , h_1) ~dh_1 \\
  & \equiv - \Delta_y \int_{h_0 - 12\Delta_y}
  ^{\Rad{-L_g}}~ \rho(L_g , h_1) ~dh_1.
\end{align*}
Differentiating again with respect to $h_0$ yields
\begin{subequations}
  \label{eq:t}
\begin{align}
  \frac{\partial^2}{\partial h_0^2} \rho(g_0, h_0) &=
  -\frac{\Delta_y}{\lambda} \Bigl( \nonumber \\
  &\quad \int_{h_0 - 12 \Delta_y}^{\Rad{-L_g}}~ \frac{\partial L_g}{\partial h_0}
  \frac{\partial}{\partial g} \rho(L_g,h_1) ~dh_1 \nonumber \\
  &\quad + \frac{ \partial \Rad{-L_g}}{\partial h_0} \rho \left(L_g,
    \Rad{-L_g} \right) \nonumber \\
  &\quad - \rho(L_g, h_0 - 12\Delta_y ) \Bigr) \nonumber \\
%
  &= -\frac{\Delta_y}{\lambda} \Bigl( \nonumber \\
  \label{eq:t1}
  &\quad \Delta_y \int_{h_0 - 12 \Delta_y}^{\Rad{-L_g}}
  \frac{\partial}{\partial g} \rho(L_g,h_1) ~dh_1  \\
  \label{eq:t2}
  &\quad - \frac{12\Delta_y}{\Rad{-L_g}} \rho \left(L_g,
    \Rad{-L_g} \right) \\
  \label{eq:t3}
  &\quad - \rho(g_0, h_0) + O(\Delta_y) \Bigr)
\end{align}
\end{subequations}
Dropping the terms that are second order in $\Delta_y$ yields
\begin{equation*}
  \frac{\partial^2}{\partial h_0^2} \rho(g_0, h_0) =
  \frac{\Delta_y}{\lambda} \rho(g_0, h_0),
\end{equation*}
which is solved by
\begin{equation*}
  \rho(g,h) = \psi(g) \left( a e^{\gamma h} + b e^{-\gamma h}
  \right),
\end{equation*}
where
\begin{equation*}
  \gamma \equiv \sqrt{\frac{\Delta_y}{\lambda}}.
\end{equation*}
From \eqref{eq:ug} and \eqref{eq:lg} the bounds $U_g$ and $L_g$ are
both $g + h\Delta_y -6\Delta_y^2$ at $h = h_{\text{max}}(g) \equiv
\Rad{-g}$.  Since the integral described in \eqref{eq:integralUL_} and
\eqref{eq:integralULq} is over an empty interval,
$\rho(g,h_{\text{max}}(g)) = 0$ and
\begin{align*}
  0 &= a e^{\gamma \Rad{-g}} + b e^{-\gamma \Rad{-g}} \\
  b &= -a e^{2\gamma \Rad{-g}} \\
  \rho(g,h) &= -\psi(g)a e^{\gamma \Rad{-g}} \left(
    e^{\gamma \left( \Rad{-g} - h \right) } -
    e^{-\gamma \left(\Rad{-g} - h \right) } \right) \\
  &= -2 \psi(g)a e^{\gamma \Rad{-g}} \sinh \left( \gamma \left( \Rad{-g}
      - h \right) \right).
\end{align*}
Choosing $a$ to cancel $-2 e^{\gamma \Rad{-g}}$, I can write the solution as
\newcommand{\rhoHG}[2]{\sinh \left( \gamma \left( \sqrt{24(u-#2)} - #1
    \right) \right)}
\begin{equation}
  \label{eq:sol_h}
  \rho(g,h) \equiv \psi(g) \rhoHG{h}{g}.
\end{equation}

The difference between the asymptotic solution \eqref{eq:sol_h} and a
numerical approximation plotted in Fig.~\ref{fig:compare} suggests an
error in the foregoing analysis.  However, notice that discarding term
\eqref{eq:t2} relies on
\begin{equation}
  \label{eq:ll}
  \Delta_y \ll \frac{\rho(g_0, h_0)}{\rho \left(L_g, \Rad{-L_g} \right)},
\end{equation}
but Fig.~\ref{fig:compare} indicates that with the given value of
$\Delta_y$ \eqref{eq:ll} fails to hold for most of the range of $h$.
The equation
\begin{equation}
  \label{eq:ugly}
  \frac{\partial^2}{\partial h^2} \rho(g, h) =
  \frac{\Delta_y}{\lambda} \rho(g, h) +
  \frac{12\Delta_y^2}{\lambda \Rad{-g}} \rho \left(g, \Rad{-g} \right)  
\end{equation}
follows from retaining \eqref{eq:t2}.  A numerical solution to
\eqref{eq:ugly} appears in \marginpar{FixMe} %Fig.~\ref{fig:ugly}.

\begin{figure}
  \centering
    \resizebox{1.0\columnwidth}{!}{
      \includegraphics{compare.pdf}}
    \caption{A comparison of the asymptotic solution \eqref{eq:sol_h}
      and a numerically estimated eigenfunction.}
  \label{fig:compare}
\end{figure}

\newcommand{\altphiHG}[2]{e^{-6\gamma^2 \Delta_y h} \sinh \left( \gamma
    \left( \sqrt{24(u-#2)} - #1 \right) \right)} I've obtained an
alternate solution by assuming the form $\rho(h,g) = f(h) \sinh\left(
  \gamma \left( \sqrt{24(u-g)} - h \right) \right)$, expanding
$\rho(h)'' = \gamma^2 \rho(h - 12\Delta_y)$ and solving for $f$ after
dropping terms that are second order in $\Delta_y$.  The alternative
solution is
\begin{equation}
  \label{eq:sol_h_}
  \rho(g,h) \equiv \psi(g) \rho(h,g) \equiv \psi(g) \altphiHG{h}{g}.  
\end{equation}

\subsection{ODE for $\psi$}
\label{sec:ODE}

Next I use \eqref{eq:sol_h} in \eqref{eq:integralULq} and
differentiate to see if a solution for $\psi(g)$ is possible.  Noting
\begin{equation*}
  \int \rho(h,g) = \frac{-1}{\gamma} \cosh \left( \gamma \left(
      \Rad{-g} - h \right) \right) + C,
\end{equation*}
I write
\begin{align}
  & \lambda \psi(g_0)\rho(h_0,g_0) \nonumber\\
  & \quad =\int_{\LgQ}^{\UgQ} \frac{-\psi(g_1) }{\gamma} \left[
    \cosh\left( \gamma \left( \Rad{-g_1} - h \right)\right) \right]_{
    \frac {\Delta_g} {\Delta_y} - 6 \Delta_y}^{\Rad{-g1}} dg_1
  \nonumber \\
  \label{eq:star}
  & \sqrt{\lambda \Delta_y} \psi(g_0) \rhoHG{h_0}{g_0} \nonumber \\
  &\quad \dhot - \int_{\Lg}^{\Ug} \psi(g_1) \left[ 1 - \cosh \left(
      \gamma \left( \sqrt{24(u-g_1)} - \frac{g_1 - g_0}{ \Delta_y}
      \right) \right) \right] dg_1.
\end{align}
Differentiating \eqref{eq:star}  wrt $g_0$ yields
\begin{align}
  & \sqrt{\lambda \Delta_y} \left[ \psi'(g_0) \rhoHG{h_0}{g_0} -
   \psi(g_0)  \frac{12 \gamma}{\sqrt{24(u-g_0)}} \cosh\left( \gamma
      \left(\sqrt{24(u-g_0)} -h_0 \right) \right) \right]  \nonumber\\
  \label{eq:d_star_dg}
  & \quad \equiv D = A + B + C \text{ where} \\
  A &= -\left( 1 - \frac{12 \Delta_y}{\sqrt{24(u-g_0)}} \right)
  \left[ \psi(g_0) + \Delta_y\sqrt{24(u-g_0)} \psi'(g_0) \right]  \nonumber\\
  & \times \left[ 1 - \cosh\left( \gamma \left(\sqrt{24(u-g_0 -
          \Delta_y \sqrt{24(u-g_0)})} - \sqrt{24(u-g_0)} \right)
    \right) \right]  \nonumber\\
  B &= \left[ \psi(g_0) + \Delta_y h_0 \psi'(g_0) \right] \left[ 1 -
    \cosh\left( \gamma \left(\sqrt{24(u-g_0 - \Delta_y h_0)} - h_0
      \right) \right) \right]  \nonumber\\
  C &= \int_{\Lg}^{\Ug} \psi(g_1) \frac{\gamma}{\Delta_y}
  \sinh \left( \gamma \left(
      \sqrt{24(u-g_1)} - \frac{g_1 - g_0 }{ \Delta_y}
    \right) \right) dg_1. \nonumber
\end{align}
If $g_0 > u - 24 \Delta_y^2$ the argument of $\cosh$ in $A$ is
complex, but if
\begin{equation}
  \label{eq:little_g}
  u-g_0 >> \Delta_y,
\end{equation}
then $A = 0 + O(\Delta_y^2)$.  I will drop $A$ assuming
\eqref{eq:little_g} and note that any solutions that I find will not
be valid for large $g$.  If $ u-g $ is large compared to $\Delta_y$,
then replacing $g_1$ with $g_0$ under the square root in $C$ might be
OK too.\marginpar{maybe not}

Integrating the approximation by parts, I find
\begin{align*}
  C &\dhot \int_{\Lg}^{\Ug} \psi(g_1) \frac{\gamma}{\Delta_y}
  \sinh \left( \gamma \left(
      \sqrt{24(u-g_0)} - \frac{g_1 - g_0 }{ \Delta_y}
    \right) \right) dg_1 \\
  &= -\left[ \psi(g_1) \cosh \left( \gamma \left(
      \sqrt{24(u-g_0)} - \frac{g_1 - g_0 }{ \Delta_y}
    \right) \right)\right]_{\Lg}^{\Ug} \\
& \quad + \int_{\Lg}^{\Ug} \psi'(g_1) \cosh \left( \gamma \left(
      \sqrt{24(u-g_0)} - \frac{g_1 - g_0 }{ \Delta_y}
    \right) \right) dg_1.
\end{align*}
The surface terms approximately cancel $B$ and thus
\begin{equation*}
  D \dhot \int_{\Lg}^{\Ug} \psi'(g_1) \cosh \left( \gamma \left(
      \sqrt{24(u-g_0)} - \frac{g_1 - g_0 }{ \Delta_y}
    \right) \right) dg_1.
\end{equation*}
Integrating $D$ by parts two more times yields (note
$\frac{\Delta_y}{\gamma} = \sqrt{\lambda \Delta_y}$)
\begin{align}
  D &\dhot -\sqrt{\lambda \Delta_y} \left[ \psi'(g_1) \sinh \left(
      \gamma \left( \sqrt{24(u-g_0)} - \frac{g_1 - g_0 }{ \Delta_y}
      \right) \right)\right]_{\Lg}^{\Ug} \nonumber \\
& \quad +\lambda \Delta_y \left[ \psi''(g_1) \cosh \left( \gamma \left(
      \sqrt{24(u-g_0)} - \frac{g_1 - g_0 }{ \Delta_y}
    \right) \right)\right]_{\Lg}^{\Ug} \nonumber \\
& \quad - \lambda \Delta_y \int_{\Lg}^{\Ug} \psi'''(g_1) \cosh \left(
  \gamma \left( \sqrt{24(u-g_0)} - \frac{g_1 - g_0 }{ \Delta_y}
  \right) \right) dg_1 \nonumber \\
\label{eq:D_end}
&\dhot \sqrt{\lambda \Delta_y} \psi'(g_0) \sinh \left( \gamma
  \left( \sqrt{24(u-g_0)} - h_0 \right) \right) \\
& \quad +\lambda \Delta_y \psi''(g_0) \left( 1 - \cosh \left( \gamma \left(
      \sqrt{24(u-g_0)} - h_o \right) \right) \right) + O\left(
  \left( \frac{\Delta_y}{\gamma} \right)^3 \right). \nonumber
\end{align}
Substituting \eqref{eq:D_end} into \eqref{eq:d_star_dg} yields
\begin{equation}
  \label{eq:hope}
  \lambda \Delta_y \psi''(g_0) \left( 1 - \cosh \left( \gamma \left(
        \sqrt{24(u-g_0)} - h_o \right) \right) \right) = -\psi(g_0)
  \frac{12 \gamma}{\sqrt{24(u-g_0)}} \cosh\left( \gamma
    \left(\sqrt{24(u-g_0)} -h_0 \right) \right),
\end{equation}
which would be a nice equation if I could find an extra
$\psi(g_0)\frac{12\gamma}{\sqrt{24(u-g_0}}$ for the right hand side.

\newpage

Differentiating \eqref{eq:star}  wrt $h_0$ yields
\newcommand{\temp}{g_0 + h_0 \Delta_y - 6 \Delta_y^2}
\newcommand{\mtemp}{-g_0 - h_0 \Delta_y + 6 \Delta_y^2}
\begin{align}
  & \frac{\lambda \gamma}{\Delta_y} \psi(g_0) \cosh \left( \gamma
    \left( \Rad{-g_0} - h_0 \right) \right) \nonumber \\
  &= - \psi\left( \temp \right)
  \left[ \cosh
    \left( \gamma
      \left( \Rad{\mtemp} - h \right)
    \right)
  \right]^{\Rad{\mtemp}}_ {\frac{h_0 \Delta_y - 6
      \Delta_y^2}{\Delta_y}} \nonumber \\
  \label{eq:DEG}
  &= \psi\left( \temp \right)
  \left[ \cosh
    \left( \gamma
      \left( \Rad{\mtemp} - h_0 + 6 \Delta_y \right)
    \right) -1
  \right].
\end{align}
The result seems implausible for two reasons that I will explain.  My
explanations assume that $\lim_{\Delta_y \rightarrow 0}
\lambda(\Delta_y) = 1$ and consequently that $\lim_{\Delta_y
  \rightarrow 0} \gamma(\Delta_y) = \sqrt{\Delta_y}$.
\begin{enumerate}
\item The ways that the variable $h_0$ appear seem to require that
  $\psi$ must depend on $h_0$, contradicting the assumption behind its
  definition.  I will assume that $h_0=0$ and later ensure that
  nonzero values are consistent with any solution that satisfies the
  second objection.
\item To lowest order in $\Delta_y$, the equation is roughly
  \begin{equation*}
    \frac{1}{\sqrt{\Delta_y}} \psi(g_0) = \psi(g_0 - 6\Delta_y^2)
    24(u-g_0) \Delta_y,
  \end{equation*}
  which suggests
  \begin{align*}
    \psi(g) &= \left( \psi(g) - 6\Delta_y^2 \psi'(g) \right) 24(u-g)
    \Delta_y^{\frac{3}{2}} \\
    \psi'(g_0) &= \frac{\Delta_y^{\frac{3}{2}} 24 (u-g_0) - 1} {144
      \Delta_y^{\frac{7}{2}} (u-g_0)} \psi(g_0) \\
    \psi'(g_0) &= \frac{- 1} {144 \Delta_y^{\frac{7}{2}} (u-g_0)} \psi(g_0)
  \end{align*}
  and solutions for $\psi(g)$ that decay with an exponent of order
  $\Delta_y^{ - \frac{7}{2}}$.  I had hoped for convergence of
  eigenfunctions as $\Delta_y \rightarrow 0$.
\end{enumerate}

I will proceed in the following two steps:
\begin{enumerate}
\item Assume $h_0=0$ and derive an ODE for $\psi$ from \eqref{eq:DEG}
  that is accurate to second order in $\Delta_y$ and fourth order
  $\gamma$.
\item Solve that ODE and with luck find $\lambda$.
\item Substitute the solution for $\psi$ into \eqref{eq:DEG} and
  ensure that at least values and first two derivatives wrt to $h_0$
  are equal.
\end{enumerate}

If $h_0 = 0$, then \eqref{eq:DEG} becomes
\renewcommand{\temp}{g_0 - 6\Delta_y^2}
\renewcommand{\mtemp}{-g_0 +6\Delta_y^2}
\begin{align*}
  & \frac{\lambda \gamma}{\Delta_y} \psi(g_0) \cosh \left( \gamma
    \left( \Rad{-g_0} \right) \right) \\
  &= \psi\left( \temp \right)
  \left[ \cosh
    \left( \gamma
      \left( \Rad{\mtemp} + 6 \Delta_y \right)
    \right) -1
  \right].
\end{align*}
To lowest order in $\Delta_y$ I find,
\newcommand{\coshg}{\cosh\left(\gamma \Rad{-g_0} \right)}
\begin{align*}
  & \frac{1}{\gamma} \psi(g_0)
  \cosh \left( \gamma \left( \Rad{-g_0} \right) \right) \\
  &= \left[\psi(g_0) - 6\Delta_y^2\psi'(g_0) \right]
  \left[ \cosh \left( \gamma \left( \Rad{-g_0} \right) \right) - 1
  \right] \\
  \psi'(g_0) &= \frac{(\gamma-1)\coshg - \gamma }{\gamma 6 \Delta_y^2
  \left( \coshg - 1 \right)} \psi(g_0).
\end{align*}
In the next sequence, I drop all but the lowest order terms in
$\gamma$ and use the first two terms in the Taylor series for
$\cosh$:
\begin{align*}
  \psi'(g_0) &\dhot -\frac{\coshg}{6\Delta_y^2 \gamma \left( \coshg -
      1 \right)} \psi(g_0) \\
  &\dhot -\frac{1}{3 \Delta_y^2 \gamma^3 24(u-g_0)} \psi(g_0) \\
  &= -\frac{1}{72 \Delta_y^2 \gamma^3 (u-g_0)} \psi(g_0).
\end{align*}
Then using the following form for solving the ODE
\begin{align*}
  y' &= \frac{by}{a-x} \text{ ODE} \\
  y(x) &= c_1 (a-x)^b \text{ solution},
\end{align*}
I find
\begin{equation}
  \label{eq:1}
  \psi(g_0) = (u - g_0)^{-\frac{1}{72 \Delta_y^2 \gamma^3}} =
  \left( \frac{1}{u - g_0} \right)^{\frac{\lambda \sqrt{\lambda}}{72
      \Delta_y^3 \sqrt{\Delta_y}}}
\end{equation}
The result is unsatisfactory because it is undefined at $g_0 = u$.

I start again.  Let $h_0=0$ and $\rho(h,g)=\rhoHG{h}{g}$, then from
\eqref{eq:integralULq}
\renewcommand{\LgQ}{g_0 - 6\Delta_y^2}
\begin{align}
  & \lambda \psi(g_0) \sinh\left(\gamma \sqrt{24(u-g_0)} \right) \nonumber \\
  &= \int_{\LgQ}^{\UgQ} \psi(g_1) \int_{\LhQ}^{\Uh}~ \rhoHG{h_1}{g_1}
  ~dh_1 dg_1 \nonumber \\
  & \lambda \gamma \psi(g_0) \sinh\left(\gamma \sqrt{24(u-g_0)}
  \right) \nonumber \\
  \label{eq:Ione}
  &= \int_{\LgQ}^{\UgQ} \psi(g_1) \left[ \cosh \left(
      \gamma \left( \sqrt{24(u-g_1)} - \frac{g_1 - g_0}{\Delta_y} + 6
        \Delta_y \right) \right) -1 \right] dg_1 \\
  &\equiv I_1 \nonumber
\end{align}
From the following change of variable
\begin{align*}
   t(g) &= \gamma \left( \sqrt{24(u-g)} - \frac{g - g_0}{\Delta_y} +
     6 \Delta_y \right)\\
   g(t) &= g_0 - 6 \Delta_y^2 - \frac{\Delta_y t}{\gamma} +
   \Delta_y \sqrt{24 \left( u-g_0 + \frac{t\Delta_y}{\gamma} \right)} \\
   \frac{d g}{ dt} &= \frac{\Delta_y}{\gamma} \left( \frac{12 \Delta_y} {
       \sqrt{ 24 \left( u - g_0 + \frac{t \Delta_y}{\gamma} \right) }}
     - 1 \right) \\
   \frac{d^2 g}{ dt^2} &= -\frac{144 \Delta_y^3}{\gamma^2}
   \left( 24
     \left( u - g_0 + \frac{t \Delta_y}{\gamma} \right)
   \right)^{-\frac{3}{2}},
\end{align*}
I calculate
\begin{align*}
  t \left( \UgQ \right) &= 0 \\
  t \left( \LgQ \right) &= \gamma \left(
    \sqrt{24\left( u - g_0 + 6 \Delta_y^2 \right) } + 12 \Delta_y \right),
\end{align*}
and write
\newcommand{\LOW}{{\gamma \left( \sqrt{24(u- g_0 + 6\Delta_y^2 )} +
      12\Delta_y \right)}}
\newcommand{\RANGE}{_\LOW ^0}
\newcommand{\INT}{\int\RANGE}
\begin{align*}
  I_1 &= \INT \psi(g(t)) \frac{d g}{d t} [\cosh(t) -1] dt \\
  &\dhot \INT \left[ \psi(g(0)) + t \psi'(g(0)) \right]
  \left[ \left. \frac{d g}{d t} \right|_{t=0}
    + t \left. \frac{d^2g}{d t^2} \right|_{t=0}
  \right]
  [\cosh(t) -1] dt \\
  &\equiv I_{A} +  I_{B} +  I_{C}
\end{align*}
\newcommand{\loit}{=_\text{loit}}
\newcommand{\DG}{\frac{\Delta_y}{\gamma} \left( \frac{12 \Delta_y}
    { \sqrt{24(u-g_0)} } - 1 \right) }
\newcommand{\DDG}{\frac{144 \Delta^2_y}{\gamma^2} \left(24(u-g_0)
  \right)^{-\frac{3}{2}} }
where (using the notation $\loit$ for \emph{low order in $t$})
\begin{align*}
  I_{A} &= \psi(g(0)) \left. \frac{d g}{d t} \right|_{t=0}
  \INT [\cosh(t) - 1] dt \\
  &= \psi(g(0)) \left. \frac{d g}{d t} \right|_{t=0}
  [\sinh(t) - t]\RANGE \\
  &\loit \psi(g(0)) \left. \frac{d g}{d t} \right|_{t=0}
  \left[\frac{t^3}{6} \right]\RANGE\\
  &= -\psi(g(0)) \DG \frac{ \left( \LOW \right)^3 }{6} \equiv \tilde
  I_A \\
  I_{B} &=  \left(
    \psi(g(0)) \left. \frac{d^2 g}{d t^2} \right|_{t=0}
    + \psi'(g(0)) \left. \frac{d g}{d t} \right|_{t=0}
  \right)
  \INT t [\cosh(t) - 1] dt \\
  &=  \left(
    \psi(g(0)) \left. \frac{d^2 g}{d t^2} \right|_{t=0}
    + \psi'(g(0)) \left. \frac{d g}{d t} \right|_{t=0}
  \right)
  \left[ t\sinh(t) - \cosh(t) - \frac{t^2}{2} \right] \RANGE \\
  &\loit  \left(
    \psi(g(0)) \left. \frac{d^2 g}{d t^2} \right|_{t=0}
    + \psi'(g(0)) \left. \frac{d g}{d t} \right|_{t=0}
  \right)
  \left[ \frac{t^4}{8} -1 \right]\RANGE \\
  &= \left( \psi(g(0)) \DDG + \psi'(g(0)) \DG \right) \\
  &\quad \times \frac{ \left( \LOW \right)^4 }{8} \equiv \tilde I_B\\
  I_{C} &= \psi'(g(0)) \left. \frac{d^2 g}{d t^2} \right|_{t=0}
  \INT t^2 [\cosh(t) - 1 ]dt \\
  &\loit \psi'(g(0)) \left. \frac{d^2 g}{d t^2} \right|_{t=0} \left[
    \frac{ t^5 }{ 8 } \right]\RANGE
\end{align*}

The equations
\begin{align*}
  \lambda \psi(g_0) \sinh\left(\gamma \sqrt{24(u-g_0)} \right) &= I_A
  + I_B \\
  \lambda \psi(g_0) \sinh\left(\gamma \sqrt{24(u-g_0)} \right) &=
  \tilde I_A + \tilde I_B
\end{align*}
are each linear first order ODEs for $\psi$ which a better man than I
might solve.

I used the following integrals copied from Wikipedia
\begin{align*}
  \int \sinh (ax)\,dx &= a^{-1} \cosh (ax) + C \\
  \int \cosh (ax)\,dx &= a^{-1} \sinh (ax) + C \\
  \int x\cosh ax\,dx &= \frac{1}{a} x\sinh ax - \frac{1}{a^2}\cosh
  ax+C,
\end{align*} 
and this from alpha
\begin{align*}
  & \int e^{ax} \sinh(b(x+c)) \\
  &= e^{ax} \frac{a \sinh(b(c+x)) - b \cosh(b(c+x))}{(a-b)(a+b)}
\end{align*}

\end{document}

%%%---------------
%%% Local Variables:
%%% eval: (TeX-PDF-mode)
%%% eval: (setq ispell-personal-dictionary "./localdict")
%%% End:
